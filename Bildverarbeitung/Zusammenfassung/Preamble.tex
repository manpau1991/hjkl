\documentclass[oneside]{scrbook}

\usepackage[utf8]{inputenc}
\usepackage[ngerman]{babel}
\usepackage{microtype}

\usepackage{scrhack} % Verschiedene Hacks für die Kompatibilität anderer Pakete mit KOMA-Script.

\usepackage{libertine}
\usepackage{amsmath,amssymb,amsthm}
\usepackage[libertine]{newtxmath} % Libertine Math.
\usepackage{commath}
\usepackage{mathtools}

% Clickable in-document links
\usepackage[hidelinks]{hyperref}
% Create an appendix
\usepackage{hyperref}

\usepackage{appendix}
\usepackage{wasysym}

\usepackage[T1]{fontenc}

% Schusterjungen und Hurenkinder
\widowpenalties 3 10000 10000 100
\clubpenalties 3 10000 10000 100
\displaywidowpenalty = 10000

\usepackage[german=guillemets]{csquotes}
\usepackage{booktabs}

\usepackage{xcolor}
\usepackage[pdftex]{graphicx}
\usepackage{tikz}
\usetikzlibrary{shapes}
\usetikzlibrary{matrix}
\usepackage{hyperref}

\allowdisplaybreaks % Allows pagebreaking of aligned equations

\newtheoremstyle{myplain}% name
  {\topsep}% Space above
  {\topsep}% Space below
  {\itshape}% Body font
  {}% Indent amount
  {\bfseries\sffamily}% Theorem head font
  {}%Punctuation after theorem head
  {.5em}%Space after theorem head
  {}% theorem head spec

\newtheoremstyle{mydefinition}% name
  {\topsep}% Space above
  {\topsep}% Space below
  {\normalfont}% Body font
  {}% Indent amount
  {\bfseries\sffamily}% Theorem head font
  {}%Punctuation after theorem head
  {.5em}%Space after theorem head
  {}% theorem head spec

\newtheoremstyle{myremark}% name
  {\topsep}% Space above
  {\topsep}% Space below
  {\normalfont}% Body font
  {}% Indent amount
  {\itshape\sffamily}% Theorem head font
  {}%Punctuation after theorem head
  {.5em}%Space after theorem head
  {}% theorem head spec

\theoremstyle{myplain}
\newtheorem{proposition}{Satz}[chapter]

\theoremstyle{mydefinition}
\newtheorem{definition}[proposition]{Definition}
\newtheorem{example}[proposition]{Beispiel}
\newtheorem{remark}[proposition]{Bemerkung}

\newcommand{\N}{\mathbb{N}}
\newcommand{\R}{\mathbb{R}}
\newcommand{\Z}{\mathbb{Z}}
\newcommand{\C}{\mathbb{C}}
\newcommand{\T}{\mathbb{T}}

\DeclareMathOperator{\sinc}{sinc}
\DeclareMathOperator{\sgn}{sgn}
\DeclareMathOperator{\supp}{supp}
\DeclareMathOperator{\Median}{Median}
\DeclareMathOperator{\DFT}{DFT}
\DeclareMathOperator{\DCT2}{DCT_{II}}

\newcommand{\INT}[1]{\text{d#1}}
\newcommand{\INTM}[1]{\text{d} #1}
\newcommand{\INTMM}[2]{\text{d#1}_{#2}}


\newcommand{\TODO}[1]{{\color{red}\textbf{\textsf{TODO:}}~#1}}