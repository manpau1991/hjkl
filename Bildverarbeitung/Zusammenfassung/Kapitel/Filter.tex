\section{Filter}

\begin{definition}[Filter und Filtertypen]
Ein Filter $ F $ ist eine Abbildung von einem Signalraum in einen anderen, oder anders formuliert, 
ein Operator, der eine Folge $ c \in l(\Z) $ in $ Fc \in l(\Z) $ abbildet. Es gibt verschiedene
Kategorien von Filtern:
\begin{description}
\item [Energieerhaltender Filter]
  In diesem Fall ist $ F : l_{2}(\Z) \rightarrow l_{2}(\Z) $ so, dass
  \[
    \norm{F}_{2} \coloneqq \sup_{\norm{c}_{2} = 1} \norm{Fc}_{2} = 1.
  \]
\item [Linearer Filter]
  Der Filter ist ein linearer Operator, d.h.\
  \[
    F(\alpha c + \beta c') = \alpha Fc + \beta Fc', \qquad \alpha, \beta \in \R, 
                                                    \quad c, c' \in l(\Z).
  \]
\item [Zeitinvarianter Filter]
  Der Operator ist \emph{stationär}, d.h.\ es ist egal, ob
  \begin{itemize}
  \item das Signal zuerst in der Zeit verschoben wird und dann Operator darauf angewendet wird,
  \item oder zuerst das Signal gefiltert und das Resultat in der Zeit verschoben wird.
  \end{itemize}
  Formal:
  \[
    F(c(\bullet + k)) = (Fc)(\bullet + k), \qquad c \in l(\Z), \quad k \in \Z.
  \]
  Oder kürzer über die Kommutativität
  \[
    F\tau_{k} = \tau_{k}F.
  \]
\item [Kausaler Filter]
  Das Ergebnis des Filters zum Zeitpunkt $ k $ hängt nur von den Eingaben in der Vergangenheit
  (also $ c(j) $ mit $ j \leq k $) ab. Der Filter kann nicht in die Zukunft sehen.
\end{description}
\end{definition}

\begin{definition}[Impulsantwort]
Die Impulsantwort eines Filters $ F $ ist definiert als
\[
  f(k) \coloneqq (F \delta)(k), \qquad k \in \Z.
\]
\end{definition}

\begin{remark}[Lineare und zeitinvariante Filter]\leavevmode
\begin{itemize}
\item Lineare und zeitinvariante Filter (sog.\ \emph{LTI-Filter}) können immer realisiert 
werden als \emph{Faltungen mit der Impulsantwort}. Sei also $ c \in l(\R) $ ein gegebenes diskretes 
Signal, dann gilt
\[
  Fc = c * f,
\]
wie man mit ein wenig Rechnerei nachweisen kann: Unser gegebenes Signal lässt sich auch schreiben
als
\[
  c = \sum_{k \in \Z} c(k) \ \delta(\bullet - k) = \sum_{k \in \Z} c(k) \ \tau_{-k} \ \delta
\]
und unter Verwendung der Linearität und Zeitinvarianz ergibt sich schließlich
\begin{align*}
   Fc
&= F \left( \sum_{k \in \Z} c(k) \ \tau_{-k} \ \delta \right) 
 = \sum_{k \in \Z} c(k) \ F(\tau_{-k} \ \delta)
 = \sum_{k \in \Z} c(k) \ \tau_{-k} \ F \delta
 = \sum_{k \in \Z} c(k) \ \tau_{-k} \ f \\
&= \sum_{k \in \Z} c(k) \ f(\bullet - k) = c * f.
\end{align*}
Das bedeutet, ein LTI-Filter lässt sich vollständig durch seine Impulsantwort charakterisieren.
\item Das Tolle daran ist: Dank der Faltung lassen sich LTI-Filter effizient am Rechner über eine 
Multiplikation der Fourier-Transformierten des Filters, der sog.\ \emph{Transferfunktion}, mit
der Fourier-Transformierten des Signals implementieren. Konkret:
\[
  Fc = \left( (Fc)^{\wedge} \right)^{\vee} = \left( (f * c)^{\wedge} \right)^{\vee}
     = \left( \widehat{f} \cdot \widehat{c} \right)^{\vee}
\]
ist sehr schnell! Denn eine Multiplikation ist wesentlich schneller und vor allem auch numerisch
stabiler durchzuführen als eine Faltung. Weiteren Zeitgewinn kann man durch Vorberechnen der
Fourier-Transformation der Impulsantwort herausschlagen. Und nicht zuletzt ist die Darstellung im
Frequenzbereich nützlich (Filterdesign funktioniert so).
\end{itemize}
\end{remark}

Von jetzt an soll die Bezeichnung \emph{digitaler Filter} immer für einen LTI-Filter stehen.

\begin{definition}[Weitere Filtertypen]
Ein digitaler Filter $ F $ heißt
\begin{enumerate}
\item \emph{FIR-Filter} (Finite Impulse Response, Filter mit endlicher Impulsantwort), wenn die
  Impulsantwort endlichen Träger hat, d.h.\ $ F\delta \in l_{00}(\Z) $.
\item \emph{IIR-Filter} (Infinite Impulse Response, Filter mit unendlicher Impulsantwort), wenn die
  Impulsantwort keinen endlichen Träger hat.
\end{enumerate}
\end{definition}

\begin{example}[FIR-Filter in der Praxis]
In der Praxis sind kausale, digitale FIR-Filter weit verbreitet. Was eigentlich nicht sehr 
verwunderlich ist, wenn man bedenkt, dass man ohne all diese Eigenschaften den Filter gar nicht
realisieren könnte.
\TODO{Hier gehört dann das Schaltbild erklärt}
\end{example}