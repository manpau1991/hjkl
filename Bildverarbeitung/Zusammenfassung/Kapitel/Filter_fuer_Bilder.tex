\section{Filter für Bilder}

\begin{definition}[Bild]
Ein Bild ist ein \emph{zweidimensionales} Signal, also eine Funktion von $ \R^{2} \rightarrow \R $
bzw.\ von $ \Z^{2} \rightarrow \R $.
\end{definition}

\begin{definition}[Signalklassen]
Mit $ l(\Z^{2}) $ bezeichnen wir die Menge aller Signale der Form
\[
  c = \left( c(\alpha) : \alpha \in \Z^{2} \right) 
    = \left( c(\alpha_{1}, \alpha_{2}) : \alpha_{1}, \alpha_{2} \in \Z \right).
\]
Den Index $ \alpha $ nennt man auch \emph{Multiindex}.

Auch für Bilder lassen sich die Normen
\[
  \norm{c}_{p} \coloneqq \left( \sum_{\alpha \in \Z^{2}} |c(\alpha)|^{p} \right)^{1/p}, \qquad
  \norm{c}_{\infty} \coloneqq \sup_{\alpha \in \Z^{2}} |c(\alpha)|
\]
für den Fall, dass $ c \in l(\R^{2}) $, bzw.
\[
  \norm{f}_{p} \coloneqq \left( \int_{\R^{2}} |f(t)|^{p} \dif t \right)^{1/p}, \qquad
  \norm{f}_{\infty} \coloneqq \sup_{x \in \R^{2}} |f(x)|
\]
falls $ f \colon \R^{2} \rightarrow \R $, definieren.
\end{definition}

\begin{remark}[Formeln für Bilder]
Der ganze Kram, den wir in den vorherigen Abschnitten für eindimensionale Signale definiert haben,
lässt sich nun sehr leicht auf das zweidimensionale übertragen (meistens muss man bei den Formeln
einfach nur $ \R $ durch $ \R^{2} $ ersetzen und hier und da eine Variable transponieren). Seien im 
Folgenden $ f, g $ zwei Bilder (kontinuierlich modelliert).
\begin{description}
\item [Fourier-Transformation] 
  Für ein $ \xi \in \R^{2} $ ist
  \[
      \widehat{f}(\xi) 
    = \int_{\R^{2}} f(t) e^{-i \xi^{\top} t} \dif t
    = \int_{\R^{2}} f(t) e^{-i (\xi_{1}t_{1} + \xi_{2}t_{2})} \dif t
    = \int_{\R^{2}} f(t) e^{-i\xi_{1}t_{1}} e^{-i\xi_{2}t_{2}} \dif t.
  \]
\item [Faltung]
  Die Faltung von $ f $ und $ g $ ist definiert als
  \[
    f * g = \int_{\R^{2}} f(x) g(\bullet - x) \dif x.
  \]
  Auch hier gilt die sehr nützliche Identität
  \[
    (f * g)^{\wedge}(\xi) = \widehat{f}(\xi) \cdot \widehat{g}(\xi).
  \]
\item [Inverse Fourier-Transformation]
  \[
    f = \left( \widehat{f} \right)^{\vee}
      = \int_{\R^{2}} \widehat{f}(t) e^{i  }
  \]
\item [Parseval/Plancherel]
\item [Poisson'sche Summenformel]
\end{description}
Für die Beweise sei auf die entsprechenden Aufgaben auf den Übungsblättern verwiesen.
\end{remark}

\begin{remark}[LTI-Filter für Bilder]
\end{remark}