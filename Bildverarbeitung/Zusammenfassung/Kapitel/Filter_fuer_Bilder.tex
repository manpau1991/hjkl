\section{Filter für Bilder}

\begin{definition}[Bild]
Ein Bild ist ein \emph{zweidimensionales} Signal, also eine Funktion von $ \R^{2} \rightarrow \R $
bzw.\ von $ \Z^{2} \rightarrow \R $.
\end{definition}

\begin{definition}[Signalklassen]
Mit $ l(\Z^{2}) $ bezeichnen wir die Menge aller Signale der Form
\[
  c = \left( c(\alpha) : \alpha \in \Z^{2} \right) 
    = \left( c(\alpha_{1}, \alpha_{2}) : \alpha_{1}, \alpha_{2} \in \Z \right).
\]
Den Index $ \alpha $ nennt man auch \emph{Multiindex}.

Auch für Bilder lassen sich die Normen
\[
  \norm{c}_{p} \coloneqq \left( \sum_{\alpha \in \Z^{2}} |c(\alpha)|^{p} \right)^{1/p}, \qquad
  \norm{c}_{\infty} \coloneqq \sup_{\alpha \in \Z^{2}} |c(\alpha)|
\]
für den Fall, dass $ c \in l(\R^{2}) $, bzw.
\[
  \norm{f}_{p} \coloneqq \left( \int_{\R^{2}} |f(t)|^{p} \dif t \right)^{1/p}, \qquad
  \norm{f}_{\infty} \coloneqq \sup_{x \in \R^{2}} |f(x)|
\]
falls $ f \colon \R^{2} \rightarrow \R $, definieren. Damit sind insbesondere die Signalräume wie
$ L_{1}(\R) $, $ l_{1}(\Z) $, usw.\ für Bilder wohldefiniert.
\end{definition}

\begin{remark}[Formeln für Bilder]
Der ganze Kram, den wir in den vorherigen Abschnitten für eindimensionale Signale definiert haben,
lässt sich nun sehr leicht auf das zweidimensionale übertragen (meistens muss man bei den Formeln
einfach nur $ \R $ durch $ \R^{2} $ ersetzen und hier und da eine Variable transponieren). Seien im 
Folgenden $ f, g $ zwei Bilder (kontinuierlich modelliert).
\begin{description}
\item [Fourier-Transformation] 
  Für ein $ \xi \in \R^{2} $ ist
  \[
      \widehat{f}(\xi) 
    = \int_{\R^{2}} f(t) e^{-i \xi^{\top} t} \dif t
    = \int_{\R^{2}} f(t) e^{-i (\xi_{1}t_{1} + \xi_{2}t_{2})} \dif t
    = \int_{\R^{2}} f(t) e^{-i\xi_{1}t_{1}} e^{-i\xi_{2}t_{2}} \dif t.
  \]
\item [Faltung]
  Die Faltung von $ f $ und $ g $ ist definiert als
  \[
    f * g = \int_{\R^{2}} f(x) g(\bullet - x) \dif x.
  \]
  Auch hier gilt die sehr nützliche Identität
  \[
    (f * g)^{\wedge}(\xi) = \widehat{f}(\xi) \cdot \widehat{g}(\xi).
  \]
\item [Inverse Fourier-Transformation]
  \[
      f(x) 
    = \left( \widehat{f} \right)^{\vee}(x)
    = \frac{1}{(2\pi)^{2}} \int_{\R^{2}} \widehat{f}(t) e^{it^{\top}x} \dif t
    = \frac{1}{2\pi} \int_{\R} \frac{1}{2\pi} \int_{\R} 
        \widehat{f}(t) e^{it_{1}x_{1}} \dif t_{1} \ e^{it_{2}x_{2}} \dif t_{2}.
  \]
\item [Parseval/Plancherel]
\[
    \int_{\R^{2}} f(t)g(t) \dif t
  = \frac{1}{(2\pi)^{2}} \int_{\R^{2}} \widehat{f}(\xi) \overline{\widehat{g}(\xi)} \dif \xi
\]
und insbesondere mit $ f = g $
\[
  \norm{f}_{2} = \frac{1}{2\pi} \norm{\widehat{f}}_{2}.
\]
\item [Poisson'sche Summenformel]
\[
    \sum_{k \in \Z^{2}} f(2k\pi)
  = \frac{1}{(2\pi)^{2}} \sum_{k \in \Z^{2}} \widehat{f}(k), \qquad
    \sum_{k \in \Z^{2}} f(k)
  = \sum_{k \in \Z^{2}} \widehat{f}(2k\pi).
\]
\end{description}
Für die Beweise sei auf die entsprechenden Aufgaben auf den Übungsblättern verwiesen.
\end{remark}

\begin{remark}[LTI-Filter für Bilder]\leavevmode
\begin{itemize}
\item Verwenden wir das Kronecker-Delta als
  \[
      \delta(k) 
    = \delta_{(0,0), (k_{1}, k_{2})}
    = \delta_{0, k_{1}} \delta_{0, k_{2}}
    = \begin{cases}
        1, & k_{1}, k_{2} = 0, \\
        0, & \text{sonst},
      \end{cases}
     \qquad k = (k_{1}, k_{2}) \in \R^{2},
  \]
  dann können wir auch für Filter $ F $, die auf zweidimensionalen Signalen operieren, eine 
  Impulsantwort definieren:
  \[
    f = F\delta \in l(\Z^{2}).
  \]
\item Dementsprechend kann man, falls $ F $ ein LTI-Filter ist, die Filterung eines Bildes $ c $ 
  wieder schreiben als Faltung der Impulsantwort von $ F $ mit dem Bild:
  \[
      Fc 
    = f * c
    = \sum_{\alpha \in \Z^{2}} f(\bullet - \alpha) c(\alpha)
    = \sum_{\alpha_{1} \in \Z} \sum_{\alpha_{2} \in \Z}
        f(\bullet - (\alpha_{1}, \alpha_{2})) \ c(\alpha_{1}, \alpha_{2}).
  \]
\item Diese Formel kann man noch weiter vereinfachen, wenn $ f $ ein Tensorprodukt ist, d.h.\ falls
  \[
    f = f_{1} \otimes f_{2}, \quad f(\alpha) = f_{1}(\alpha_{1}) \cdot f_{2}(\alpha_{2})
  \]
  für $ f_{1}, f_{2} \in l(\Z) $ und $ \alpha = (\alpha_{1}, \alpha_{2}) \in \R^{2} $. Denn dann
  gilt weiter
  \begin{align*}
     f * c
  &= \sum_{\alpha_{1} \in \Z} \sum_{\alpha_{2} \in \Z}
        f(\bullet - (\alpha_{1}, \alpha_{2})) \ c(\alpha_{1}, \alpha_{2}) \\
  &= \sum_{\alpha_{1} \in \Z} \sum_{\alpha_{2} \in \Z} 
        f_{1}(\bullet_{1} - \alpha_{1}) \ f_{2}(\bullet_{2} - \alpha_{2}) \ 
        c(\alpha_{1}, \alpha_{2}) \\
  &= \sum_{\alpha_{1} \in \Z} f_{1}(\bullet_{1} - \alpha_{1})
        \sum_{\alpha_{2} \in \Z} f_{2}(\bullet_{2} - \alpha_{2}) \ c(\alpha_{1}, \alpha_{2}) \\
  &= \sum_{\alpha_{1} \in \Z} f_{1}(\bullet_{1} - \alpha_{1})
        \left( f_{2} * c(\alpha_{1}, \bullet) \right).
  \end{align*}
  Das bedeutet: Ist die Impulsantwort des LTI-Filters ein Tensorprodukt, dann erfolgt die Filterung
  eines Bildes folgendermaßen: 
  \begin{enumerate}
  \item Zuerst werden die Zeilen $ c(\alpha_{1}, \bullet) $ des Bildes für jedes feste
    $ \alpha_{1} $ mit $ f_{2} $ gefiltert. Diese Operation reduziert das Bild auf einen
    einzigen Ergebnisvektor, welchen wir als Spalte auffassen.
  \item Diese Spalte wird anschließend mit $ f_{1} $ gefiltert. Das Resultat ist dann gerade 
    $ f * c $.
  \end{enumerate}
  Als grafische Veranschaulichung sei auf Abbildung~\ref{fig:FIR-Filter-2D} verwiesen. Man kann
  sich leicht überlegen, dass die Filterung der Spalten mit $ f_{1} $ zuerst und anschließende
  Filterung der Ergebniszeile mit $ f_{2} $ zum obigen Vorgehen äquivalent ist. Wenn man das Bild um
  $ 90^{\circ} $ dreht, sollte sich ja schließlich nichts ändern, oder?
\item Auch die Transferfunktion $ \widehat{f} $ lässt sich besonders elegant schreiben, falls
  $ f = f_{1} \otimes f_{2} $:
  \begin{align*}
     \widehat{f}(\xi)
  &= \sum_{\alpha \in \Z^{2}} f(\alpha) e^{-i \alpha^{\top} \xi}
   = \sum_{\alpha_{1}, \alpha_{2} \in \Z} 
       f(\alpha_{1}, \alpha_{2}) e^{-i (\alpha_{1}\xi_{1} + \alpha_{2}\xi_{2})} \\
  &= \sum_{\alpha_{1} \in \Z} \sum_{\alpha_{2} \in \Z} 
        f_{1}(\alpha_{1}) f_{2}(\alpha_{2}) e^{-i\alpha_{1}\xi_{1}} e^{-i\alpha_{2}\xi_{2}} \\
  &= \left( \sum_{\alpha_{1} \in \Z} f_{1}(\alpha_{1}) e^{-i\alpha_{1}\xi_{1}} \right)
        \left( \sum_{\alpha_{2} \in \Z} f_{2}(\alpha_{2}) e^{-i\alpha_{2}\xi_{2}} \right) \\
  &= \widehat{f_{1}}(\xi_{1}) \cdot \widehat{f_{2}}(\xi_{2}).
  \end{align*}
  Und damit wird die Anwendung des Filters besonders einfach:
  \[
      Fc
    = \left( (Fc)^{\wedge} \right)^{\vee}
    = \left( (f * c)^{\wedge} \right)^{\vee}
    = \left( \widehat{f} \cdot \widehat{c} \right)^{\vee}
    = \left( \widehat{f_{1}} \cdot \widehat{f_{2}} \cdot \widehat{c} \right)^{\vee}.
  \]
\end{itemize}
\begin{figure}[ht]
\centering
\begin{tikzpicture}
\matrix [matrix of math nodes] {
  \ddots & \vdots & \vdots & \ddots & 																		& \vdots \\
  \cdots & c(0,0) & c(0,1) & \cdots & \rightarrow f_{2} * c(0, \bullet) = & c'(0) \\
  \cdots & c(1,0) & c(1,1) & \cdots & \rightarrow f_{2} * c(1, \bullet) = & c'(1) \\
  \ddots & \vdots & \vdots & \ddots & 																		& \vdots \\
         &			  &				 & 				&																			& \downarrow \\
         &			  &				 & 				&																			& f_{1} * c' \\
         &			  &				 & 				&																			& \downarrow \\
         &			  &				 & 				&																			& f * c \\
};
\end{tikzpicture}
\caption{Schematische Darstellung der Funktionsweise eines FIR-Filters für Bilder, wenn dessen
  Impulsantwort eine Tensorproduktstruktur aufweist. Zuerst wird das Bild horizontal gefiltert
  und so zu einer Spalte reduziert, welche dann vertikal gefiltert wird, um so das Endergebnis zu
  erhalten.}
\label{fig:FIR-Filter-2D}
\end{figure}
\end{remark}

\begin{example}[Filter in der Bildverarbeitung]
Wir betrachten im Folgenden ein paar Beispiele für bekannte Filter in der Bildverarbeitung. Dabei
ist es meistens so, dass die Filter für Signale der Form $ \phi \colon \R^{2} \rightarrow \R $, 
also kontinuierliche Daten, konzipiert sind und später erst diskretisiert werden.
\begin{description}
\item [Mittelwertfilter] Der Mittelwertfilter $ F $ mit Impulsantwort
  \[
    f = \frac{1}{|\Omega|}\chi_{\Omega}, \qquad \Omega \subset \R^{2},
  \]
  filtert das Signal $ \phi $ mittels
  \[
      \phi(x) \mapsto (f * \phi)(x)
    = \frac{1}{|\Omega|} \int_{\R} \chi_{\Omega}(t) \phi(x - t) \dif t
    = \frac{1}{|\Omega|} \int_{\Omega} \phi(x - t) \dif t
    = \frac{1}{|\Omega|} \int_{x + \Omega} \phi(t) \dif t.
  \]
  Das heißt, er ordnet $ \phi(x) $ den Mittelwert der Menge $ x + \Omega $ zu. Um den Filter zu
  diskretisieren, setzt man einfach $ f = S_{h}\chi_{\Omega} $ und normalisiert dann, indem man 
  durch die Anzahl der in $ \Omega $ enthaltenen Abtastpunkte teilt.
  
  Ein großer Vorteil des Mittelwertfilters ist, dass er Rauschen unterdrücken kann. Dazu modelliert
  man das gemessene Signal als $ \phi(x) = \psi(x) + \epsilon(x) $, wobei $ \psi $ die eigentlichen
  Daten sind und $ \epsilon $ das Rauschen. Unter der Annahme, dass das Rauschen mittelwertfrei ist,
  erhält man so
  \[
      F\phi (x)
    = \frac{1}{|\Omega|} \int_{x + \Omega} \psi(t) \dif t
        + \frac{1}{|\Omega|} \int_{x + \Omega} \epsilon(t) \dif t.
  \]
  
  Das Problem beim Design eines Mittelwertfilters ist die korrekte Wahl von $ \Omega $. Denn der 
  Filter angewandt auf die eigentlichen Daten $ \psi $ sollte diese weitestgehend unberührt lassen, 
  d.h.\ $ F\psi \sim \psi $. Dafür muss $ \Omega $ eine möglichst kleine Menge sein. Gleichzeitig 
  soll aber auch das Rauschen möglichst gut ausgemittelt werden, und dafür muss $ \Omega $ groß
  genug sein.
  
  Eine andere Möglichkeit, einen Mittelwertfilter zu realisieren, ist
  \[
    f(x_{1},x_{2}) 
      = \frac{1}{2\pi\sigma^{2}} \exp\left( -\frac{x_{1}^{2} + x_{2}^{2}}{2\sigma^{2}} \right),
      \qquad \sigma > 0
  \]
  zu setzten. Für die Diskretisierung tastet man $ f\chi_{[-N,N]^{2}} $ für ein $ N \in N $ ab. Die
  charakteristische Funktion wird benötigt, um einen FIR-Filter zu erhalten, denn der Gaußkern
  besitzt ja unendlichen Träger.
\item [Binomialfilter]
  Die Binomalfilter sind spezielle Mittelwertfilter, denen wir in einem eigenen Punkt 
  Aufmerksamkeit widmen wollen, und das diskrete Gegenstück zum oben erwähnten Gaußkern. Deren 
  Impulsantwort ist gegeben durch
  \[
    f(j,k) = \frac{1}{2^{m + n}} \binom{m}{j} \binom{m}{k},
      \qquad 0, \ldots, m, \quad 0, \ldots, n.
  \]
  Die Impulsantwort lässt sich auch als Matrix darstellen, wenn man $ j $ als Zeile und $ k $ als
  Spalte auffasst. Beispielsweise erhält man für $ m,n = 2 $ die Matrix
  \[
    \frac{1}{16} \begin{pmatrix}
    1 & 2 & 1 \\
    2 & 4 & 2 \\
    1 & 2 & 1
    \end{pmatrix}
  \]
  und für $ m,n=4 $
  \[
    \frac{1}{256} \begin{pmatrix}
    1 &  4 & 6  & 4  & 1 \\
    4 & 16 & 24 & 16 & 4 \\
    6 & 24 & 36 & 24 & 6 \\
    4 & 16 & 24 & 16 & 4 \\
    1 &  4 &  6 &  4 & 1
    \end{pmatrix}.
  \]
  Übrigens ist $ 2^{m + n} $ gerade die Summe aller Einträge in der Matrix, denn es gilt
  \begin{align*}
      \sum_{j = 0}^{m} \sum_{k = 0}^{n} \binom{m}{j} \binom{n}{k}
   &= \left( \sum_{j = 0}^{m} \binom{m}{j} \right) \left( \sum_{k = 0}^{n} \binom{n}{k} \right)
    = \left( \sum_{j = 0}^{m} \binom{m}{j} 1^{m} 1^{m - j} \right) 
        \left( \sum_{k = 0}^{n} \binom{n}{k} 1^{n} 1^{n - k} \right) \\
   &= (1 + 1)^{m} (1 + 1)^{n}
    = 2^{m} 2^{n}
    = 2^{m + n},
  \end{align*}
  sodass wir es wirklich mit einem Mittelwertfilter zu tun haben!
\end{description}
\TODO{Binomialfilter, Gradientenfilter, Laplacefilter}
\end{example}