\section{Filter für Bilder}

\begin{definition}[Bild]
Ein Bild ist ein \emph{zweidimensionales} Signal, also eine Funktion von $ \R^{2} \rightarrow \R $
bzw.\ von $ \Z^{2} \rightarrow \R $.
\end{definition}

\begin{definition}[Signalklassen]
Mit $ l(\Z^{2}) $ bezeichnen wir die Menge aller Signale der Form
\[
  c = \left( c(\alpha) : \alpha \in \Z^{2} \right) 
    = \left( c(\alpha_{1}, \alpha_{2}) : \alpha_{1}, \alpha_{2} \in \Z \right).
\]
Den Index $ \alpha $ nennt man auch \emph{Multiindex}.

Auch für Bilder lassen sich die Normen
\[
  \norm{c}_{p} \coloneqq \left( \sum_{\alpha \in \Z^{2}} |c(\alpha)|^{p} \right)^{1/p}, \qquad
  \norm{c}_{\infty} \coloneqq \sup_{\alpha \in \Z^{2}} |c(\alpha)|
\]
für den Fall, dass $ c \in l(\R^{2}) $, bzw.
\[
  \norm{f}_{p} \coloneqq \left( \int_{\R^{2}} |f(t)|^{p} \dif t \right)^{1/p}, \qquad
  \norm{f}_{\infty} \coloneqq \sup_{x \in \R^{2}} |f(x)|
\]
falls $ f \colon \R^{2} \rightarrow \R $, definieren. Damit sind insbesondere die Signalräume wie
$ L_{1}(\R) $, $ l_{1}(\Z) $, usw.\ für Bilder wohldefiniert.
\end{definition}

\begin{remark}[Formeln für Bilder]
Der ganze Kram, den wir in den vorherigen Abschnitten für eindimensionale Signale definiert haben,
lässt sich nun sehr leicht auf das zweidimensionale übertragen (meistens muss man bei den Formeln
einfach nur $ \R $ durch $ \R^{2} $ ersetzen und hier und da eine Variable transponieren). Seien im 
Folgenden $ f, g $ zwei Bilder (kontinuierlich modelliert).
\begin{description}
\item [Fourier-Transformation] 
  Für ein $ \xi \in \R^{2} $ ist
  \[
      \widehat{f}(\xi) 
    = \int_{\R^{2}} f(t) e^{-i \xi^{\top} t} \dif t
    = \int_{\R^{2}} f(t) e^{-i (\xi_{1}t_{1} + \xi_{2}t_{2})} \dif t
    = \int_{\R^{2}} f(t) e^{-i\xi_{1}t_{1}} e^{-i\xi_{2}t_{2}} \dif t.
  \]
\item [Faltung]
  Die Faltung von $ f $ und $ g $ ist definiert als
  \[
    f * g = \int_{\R^{2}} f(x) g(\bullet - x) \dif x.
  \]
  Auch hier gilt die sehr nützliche Identität
  \[
    (f * g)^{\wedge}(\xi) = \widehat{f}(\xi) \cdot \widehat{g}(\xi).
  \]
\item [Inverse Fourier-Transformation]
  \[
      f(x) 
    = \left( \widehat{f} \right)^{\vee}(x)
    = \frac{1}{(2\pi)^{2}} \int_{\R^{2}} \widehat{f}(t) e^{it^{\top}x} \dif t
    = \frac{1}{2\pi} \int_{\R} \frac{1}{2\pi} \int_{\R} 
        \widehat{f}(t) e^{it_{1}x_{1}} \dif t_{1} \ e^{it_{2}x_{2}} \dif t_{2}.
  \]
\item [Parseval/Plancherel]
\[
    \int_{\R^{2}} f(t)g(t) \dif t
  = \frac{1}{(2\pi)^{2}} \int_{\R^{2}} \widehat{f}(\xi) \overline{\widehat{g}(\xi)} \dif \xi
\]
und insbesondere mit $ f = g $
\[
  \norm{f}_{2} = \frac{1}{(2\pi)^{2}} \norm{\widehat{f}}_{2}.
\]
\item [Poisson'sche Summenformel]
\[
    \sum_{k \in \Z^{2}} f(2k\pi)
  = \frac{1}{(2\pi)^{2}} \sum_{k \in \Z^{2}} \widehat{f}(k), \qquad
    \sum_{k \in \Z^{2}} f(k)
  = \sum_{k \in \Z^{2}} \widehat{f}(2k\pi).
\]
\end{description}
Für die Beweise sei auf die entsprechenden Aufgaben auf den Übungsblättern verwiesen.
\end{remark}

\begin{remark}[LTI-Filter für Bilder]\leavevmode
\begin{itemize}
\item Verwenden wir das Kronecker-Delta als
  \[
    \delta(k) = \delta(k_{1}) \delta({k_{2}}), \qquad k = (k_{1}, k_{2}) \in \R^{2},
  \]
  dann können wir auch für Filter $ F $, die auf zweidimensionalen Signalen operieren, eine 
  Impulsantwort definieren:
  \[
    f = F\delta \in l(\Z^{2}).
  \]
\item Dementsprechend kann man, falls $ F $ ein LTI-Filter ist, die Filterung eines Bildes $ c $ 
  wieder schreiben als Faltung der Impulsantwort von $ F $ mit dem Bild:
  \[
      Fc 
    = f * c
    = \sum_{\alpha \in \Z^{2}} f(\bullet - \alpha) c(\alpha)
    = \sum_{\alpha_{1} \in \Z} \sum_{\alpha_{2} \in \Z}
        f(\bullet - (\alpha_{1}, \alpha_{2})) \ c(\alpha_{1}, \alpha_{2}).
  \]
\item Diese Formel kann man noch weiter vereinfachen, wenn $ f $ ein Tensorprodukt ist, d.h.\ falls
  \[
    f = f_{1} \otimes f_{2}, \quad f(\alpha) = f_{1}(\alpha_{1}) \cdot f_{2}(\alpha_{2})
  \]
  für $ f_{1}, f_{2} \in l(\Z) $ und $ \alpha = (\alpha_{1}, \alpha_{2}) \in \R^{2} $. Denn dann
  gilt weiter
  \begin{align*}
     f * c
  &= \sum_{\alpha_{1} \in \Z} \sum_{\alpha_{2} \in \Z}
        f(\bullet - (\alpha_{1}, \alpha_{2})) \ c(\alpha_{1}, \alpha_{2}) \\
  &= \sum_{\alpha_{1} \in \Z} \sum_{\alpha_{2} \in \Z} 
        f_{1}(\bullet_{1} - \alpha_{1}) \ f_{2}(\bullet_{2} - \alpha_{2}) \ 
        c(\alpha_{1}, \alpha_{2}) \\
  &= \sum_{\alpha_{1} \in \Z} f_{1}(\bullet_{1} - \alpha_{1})
        \sum_{\alpha_{2} \in \Z} f_{2}(\bullet_{2} - \alpha_{2}) \ c(\alpha_{1}, \alpha_{2}) \\
  &= \sum_{\alpha_{1} \in \Z} f_{1}(\bullet_{1} - \alpha_{1})
        \left( f_{2} * c(\alpha_{1}, \bullet) \right).
  \end{align*}
  Das bedeutet: Ist die Impulsantwort des LTI-Filters ein Tensorprodukt, dann erfolgt die Filterung
  eines Bildes folgendermaßen: 
  \begin{enumerate}
  \item Zuerst werden die Zeilen $ c(\alpha_{1}, \bullet) $ des Bildes für jedes feste
    $ \alpha_{1} $ mit $ f_{2} $ gefiltert. Diese Operation reduziert das Bild auf einen
    einzigen Ergebnisvektor, welchen wir als Spalte auffassen.
  \item Diese Spalte wird anschließend mit $ f_{1} $ gefiltert. Das Resultat ist dann gerade 
    $ f * c $.
  \end{enumerate}
  Als grafische Veranschaulichung sei auf Abbildung~\ref{fig:FIR-Filter-2D} verwiesen.
\item Auch die Transferfunktion $ \widehat{f} $ lässt sich besonders elegant schreiben, falls
  $ f = f_{1} \otimes f_{2} $:
  \begin{align*}
     \widehat{f}(\xi)
  &= \sum_{\alpha \in \Z^{2}} f(\alpha) e^{-i \alpha^{\top} \xi}
   = \sum_{\alpha \in \Z^{2}} f(\alpha) e^{-i (\alpha_{1}\xi_{1} + \alpha_{2}\xi_{2})} \\
  &= \sum_{\alpha_{1} \in \Z} \sum_{\alpha_{2} \in \Z} 
        f_{1}(\alpha_{1}) f_{2}(\alpha_{2}) e^{-i\alpha_{1}\xi_{1}} e^{-i\alpha_{2}\xi_{2}} \\
  &= \left( \sum_{\alpha_{1} \in \Z} f_{1}(\alpha_{1}) e^{-i\alpha_{1}\xi_{1}} \right)
        \left( \sum_{\alpha_{2} \in \Z} f_{2}(\alpha_{2}) e^{-i\alpha_{2}\xi_{2}} \right) \\
  &= \widehat{f_{1}}(\xi_{1}) \cdot \widehat{f_{2}}(\xi_{2}).
  \end{align*}
  Und damit wird die Anwendung des Filters besonders einfach:
  \[
      Fc
    = \left( (Fc)^{\wedge} \right)^{\vee}
    = \left( (f * c)^{\wedge} \right)^{\vee}
    = \left( \widehat{f} \cdot \widehat{c} \right)^{\vee}
    = \left( \widehat{f_{1}} \cdot \widehat{f_{2}} \cdot \widehat{c} \right)^{\vee}.
  \]
\end{itemize}
\begin{figure}[ht]
\centering
\begin{tikzpicture}
\matrix [matrix of math nodes] {
  \ddots & \vdots & \vdots & \ddots & 																		& \vdots \\
  \cdots & c(0,0) & c(0,1) & \cdots & \rightarrow f_{2} * c(0, \bullet) = & c'(0) \\
  \cdots & c(1,0) & c(1,1) & \cdots & \rightarrow f_{2} * c(1, \bullet) = & c'(1) \\
  \ddots & \vdots & \vdots & \ddots & 																		& \vdots \\
         &			  &				 & 				&																			& \downarrow \\
         &			  &				 & 				&																			& f_{1} * c' \\
         &			  &				 & 				&																			& \downarrow \\
         &			  &				 & 				&																			& f * c \\
};
\end{tikzpicture}
\caption{Schematische Darstellung der Funktionsweise eines FIR-Filters für Bilder, wenn dessen
  Impulsantwort eine Tensorproduktstruktur aufweist. Zuerst wird das Bild horizontal gefiltert
  und so zu einer Spalte reduziert, welche dann vertikal gefiltert wird, um so das Endergebnis zu
  erhalten.}
\label{fig:FIR-Filter-2D}
\end{figure}
\end{remark}

\begin{example}[Filter in der Bildverarbeitung]
\TODO{Mittelwertfilter, Binomialfilter, Gradientenfilter, Laplacefilter}
\end{example}