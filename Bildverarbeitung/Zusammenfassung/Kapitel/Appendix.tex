
\section{Dichtheit von $L_p$-Räumen}\label{seq:densityoflebesguespaces}
Wir zeigen nun, dass speziell für unsere Zwecke gilt, dass
$ L_1(\R) \cap L_2(\R) $ dicht in $L_2(\R)$ liegt.

\begin{proof}
Sei $f\in L_1(\R) \cap L_2(\R)$.

Dann ist $f$ auch aus $L_1(\R)$ (da es im Schnitt liegt). Das Lemma von
\emph{Riemann-Lebesgue} liefert uns, dass $f$ dann im Unendlichen verschwindet. Bei
integrablen Funktionen (also Funktionen aus unseren Signalräumen) ist die Fourier-Transformation
außerdem stetig. D.h., $f$ ist eine $C_0$-Funktion.

Es gilt, dass $C_0$ dicht in $L_2$ und dicht in $L_1$ liegt\footnote{Dies ist ein Resultat, welches
ich im Netz gefunden habe und hier einfach so verwende.}. Da dies für beide Räume gilt, so liegt
$C_0$ auch dicht in $L_1 \cap L_2$.

Mit den Ergebnissen, dass $C_0 \subset L_2$ dicht, $C_0 \subset L_1 \cap L_2$ dicht und 
$L_1 \cap L_2 \subset L_2$ ist, so folgt, dass $L_1 \cap L_2$ dicht in $L_2$ liegt.
\end{proof}

\section{Sinus Cardinalis}\label{sec:sincproofs}

\subsection{Integrierbarkeit}
Es wird gezeigt, dass die $ \sinc $-Funktion nicht integrierbar ist.
\begin{proof}
Wir stellen zunächst fest, dass der $ \sinc $ symmetrisch um die $ y $-Achse ist. Außerdem besitzt
er seine Nullstellen nur an den ganzzahligen Werten. Dies rechtfertigt
\[
    \norm{\sinc}_{1}
  = \int_{\R} |\sinc(x)| \dif x
  = 2 \int_{0}^{\infty} |\sinc(x)| \dif x
  = 2 \sum_{k = 0}^{\infty} \int_{k}^{k + 1} |\sinc(x)| \dif x.
\]
Wir schätzen das Integral
\begin{align*}
    \int_{k}^{k + 1} |\sinc(x)| \dif x
 &= \int_{k}^{k + 1} \left| \frac{\sin(\pi x)}{\pi x} \right|  \dif x
  = \int_{k}^{k + 1} \frac{|\sin(\pi x)|}{\pi x} \dif x \\
 &\geq \int_{k}^{k + 1} \frac{|\sin(\pi x)|}{\pi(k + 1)} \dif x
\end{align*}
nach unten ab und erhalten mit der Substitution $ t \coloneqq \pi x $, $ \dif t = \pi \dif x $
\begin{align*}
   \int_{k}^{k + 1} \frac{|\sin(\pi x)|}{\pi(k + 1)} \dif x
&= \frac{1}{\pi^{2}(k + 1)} \int_{k\pi}^{(k + 1)\pi} |\sin(t)| \dif t \\
&= \frac{1}{\pi^{2}(k + 1)} 
      \eval[3]{ \left( -\cos(x) \sgn(\sin(x) \right) }_{t = k\pi}^{(k + 1)\pi} \\
&= \frac{1}{\pi^{2}(k + 1)} (\underbrace{\cos(k\pi) + \cos((k + 1)\pi)}_{= 1 + 1 = 2})\\
&= \frac{2}{\pi^{2}(k + 1)}.
\end{align*}
Die Reihe
\[
    \sum_{k = 0}^{\infty} \frac{2}{\pi^{2}(k + 1)}
  = \frac{2}{\pi^{2}} \sum_{k = 1}^{\infty} \frac{1}{k}
  > \infty
\]
divergiert und damit ist auch $ \norm{\sinc}_{1} > \infty $, also $ \sinc \notin L_{1}(\R) $.
\end{proof}
Auf ähnliche Weise lässt sich übrigens zeigen, dass $ \sinc \in L_{2}(\R) $.

\subsection{Flächeninhalt}
Gesucht ist der Wert des Integrals
\[
  \int_{\R} \sinc(x) \dif x,
\]
also der Flächeninhalt, der zwischen der $ x $-Achse und der $ \sinc $-Funktion eingeschlossen wird.
Dazu machen wir uns zuerst klar, dass für beliebige $ x \in \R \setminus \{ 0 \} $ gilt
\[
    \int_{0}^{\infty} e^{-xt} \dif t 
  = \eval{ \left( -\frac{e^{-xt}}{x} \right) }_{t = 0}^{\infty}
  = 0 - \left( -\frac{1}{x} \right) 
  = \frac{1}{x}.
\]
Danach verwenden wir den Satz von Fubini, um zu erhalten:
\[
    \int_{\R} \frac{\sin(x)}{x} \dif x
  = 2 \int_{0}^{\infty} \frac{\sin(x)}{x} \dif x
  = 2 \int_{0}^{\infty} \int_{0}^{\infty} e^{-xt} \sin(x) \dif x \dif t.
\]
Partielle Integration liefert uns
\begin{align*}
   \int_{0}^{\infty} e^{-xt} \sin(x) \dif x 
&= \eval{ \left( -\frac{e^{-xt}}{t} \sin(x) \right) }_{x = 0}^{\infty}
      - \int_{0}^{\infty} -\frac{e^{-xt}}{t} \cos(x) \dif x \\
&= 0 + 0 + \int_{0}^{\infty} \frac{e^{-xt}}{t} \cdot \frac{e^{ix} + e^{-ix}}{2} \dif x \\
&= \frac{1}{2t} \left( 
      \int_{0}^{\infty} e^{-x(t - i)} \dif x + \int_{0}^{\infty} e^{-x(t + i)} \dif x
   \right) \\
&= \frac{1}{2t} \left(
      \eval{ - \frac{e^{-x(t-i)}}{t-i} }_{x = 0}^{\infty}
      + \eval{ \frac{e^{-x(t-i)}}{t+i} }_{x = 0}^{\infty}
   \right) \\
&= \frac{1}{2t} \left( 0 + \frac{1}{t - i} + 0 + \frac{1}{t + i} \right)
 = \frac{1}{2t} \cdot \frac{2t}{t^{2} + 1} \\
&= \frac{1}{t^{2} + 1}.
\end{align*}
Folglich ist
\[
   \int_{\R} \frac{\sin(x)}{x} \dif x
= 2 \int_{0}^{\infty} \frac{1}{t^{2} + 1} \dif t
= 2 \eval[2]{\arctan(t)}_{t=0}^{\infty}
= 2 \left( \frac{\pi}{2} - 0 \right)
= \pi.
\]
Schließlich erhält man
\[
    \int_{\R} \sinc(x) \dif x
  = \int_{\R} \frac{\sin(\pi x)}{\pi x} \dif x
  = \frac{1}{\pi} \int_{\R} \frac{\sin(t)}{t} \dif t
  = \frac{1}{\pi} \cdot \pi
  = 1
\]
mit Substitution von $ t \coloneqq \pi x $, $ \dif t = \pi \dif x $.