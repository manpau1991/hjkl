
\section{Dichtheit von $L_p$-Räumen}\label{seq:densityoflebesguespaces}
Wir zeigen nun, dass speziell für unsere Zwecke gilt, dass
$ L_1(\R) \cap L_2(\R) $ dicht in $L_2(\R)$ liegt.

\begin{proof}
Sei $f\in L_1(\R) \cap L_2(\R)$.

Dann ist $f$ auch aus $L_1(\R)$ (da es im Schnitt liegt). Das Lemma von
\emph{Riemann-Lebesgue} liefert uns, dass $f$ dann im Unendlichen verschwindet. Bei
integrablen Funktionen (also Funktionen aus unseren Signalräumen) ist die Fourier-Transformation
außerdem stetig. D.h., $f$ ist eine $C_0$-Funktion.

Es gilt, dass $C_0$ dicht in $L_2$ und dicht in $L_1$ liegt\footnote{Dies ist ein Resultat, welches
ich im Netz gefunden habe und hier einfach so verwende.}. Da dies für beide Räume gilt, so liegt
$C_0$ auch dicht in $L_1 \cap L_2$.

Mit den Ergebnissen, dass $C_0 \subset L_2$ dicht, $C_0 \subset L_1 \cap L_2$ dicht und 
$L_1 \cap L_2 \subset L_2$ ist, so folgt, dass $L_1 \cap L_2$ dicht in $L_2$ liegt.
\end{proof}




