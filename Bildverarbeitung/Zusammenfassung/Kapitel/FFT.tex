\section{Die Schnelle Fourier-Transformation (FFT)}

Wir kommen nun zum Heiligen Gral in der Bildverarbeitung, nämlich der sogenannten \emph{Schnellen 
Fourier-Transformation (FFT)}. Da es sich dabei im um eine schnelle Implementierung der Diskreten 
Fourier-Transformation (DFT) handelt, wollen wir uns diese zunächst etwas genauer ansehen.

\subsection{Die Diskrete Fourier-Transformation (DFT)}

Wir werden nun kurz die Formel der Diskreten Fourier-Transformation herleiten. Die 
(kontinuierliche) Fourier-Transformation einer Folge $ c \in l_{1}(\R) $ ist eine
$ 2\pi $-periodische Funktion $ \widehat{c} \in L_{1}(\T) $. Die einfachste Möglichkeit
$ \widehat{c} $ zu diskretisieren, ist $ \widehat{c} $ einfach an allen $ 2\pi / n $-ten Stellen 
für ein vorher gewähltes $ n \in \N $ abzutasten. So erhalten wir
\[
    \DFT_{n}
  \coloneq \widehat{c}_{n}
  \coloneq S_{2\pi / n} \ \widehat{c}
  = S_{2\pi / n} \ \sum_{k \in \Z} c(k) e^{-ik\bullet}
  = \sum_{k \in \Z} c(k) \ S_{2\pi / n} \ e^{-ik\bullet}
  = \sum_{k \in \Z} c(k) e^{-2\pi ik\bullet / n}.
\]

\begin{definition}[Diskrete Fourier-Transformation]
Zu einer Folge $ c \in l_{1}(\Z) $ bezeichnet man
\[
  \widehat{c}_{n} \coloneq \sum_{k \in \Z} c(k) e^{-2\pi ik\bullet / n}, \qquad n \in \N
\]
als die \emph{Diskrete Fourier-Transformation} (DFT) der Ordnung $ n $ von $ c $.
\end{definition}

\begin{remark}\leavevmode
\begin{itemize}
\item Die DFT $ \widehat{c}_{n} $ ist wegen
  \[
      \widehat{c}_{n}(\bullet + n)
    = \sum_{k \in \Z} c(k) e^{-2\pi ik(\bullet + n) / n}
    = \sum_{k \in \Z} c(k) e^{-2\pi ik(\bullet / n + 1)}
    = \sum_{k \in \Z} c(k) e^{-2\pi ik\bullet / n}
  \]
  $ n $-periodisch. Das heißt, sie ist durch die Werte
  \[
    \widehat{c}_{n}(k), \qquad k \in \Z_{n} = \Z / n\Z \simeq \{0, 1, \ldots, n - 1\}
  \]
  eindeutig festgelegt.
\item Für $ m $-periodische oder $ m $-periodisierte Folgen $ c \in l(\Z_{m}) $ lässt sich die DFT 
  der Ordnung $ n $ auch als Matrix darstellen:
  \[
    \widehat{c}_{n} = V_{n,m}c, \qquad
    V_{n,m} \coloneq \left( e^{-2\pi i j k / n} : j \in \Z_{n}, k \in Z_{m} \right).
  \]
  Die Matrix $ V_{n,m} $ bezeichnen wir auch als \emph{Transformationsmatrix}.
\item Ist $ m = n $, so schreiben wir statt $ V_{n,n} $ einfach $ V_{n} $. In diesem Fall definieren
  wir die primitive $ n $-te Einheitswurzel
  \[
    \omega \coloneq e^{-2\pi i / n}, \qquad \omega^{n} = e^{-2\pi i} = 1.
  \]
  Dann gilt
  \[
      V_{n} 
    = \left( \omega^{jk} : j,k \in \Z_{n} \right)
    = \begin{pmatrix}
        1 & 1 & \cdots & 1 & 1 \\
        1 & \omega & \cdots & \omega^{n - 2} & \omega^{n - 1} \\
        1 & \omega^{2} & \cdots & \omega^{2(n - 2)} & \omega^{2(n - 1)} \\
        \vdots & \vdots & \ddots & \vdots & \vdots \\
        1 & \omega^{n - 2} & \cdots & \omega^{4} & \omega^{2} \\
        1 & \omega^{n - 1} & \cdots & \omega^{2} & \omega
      \end{pmatrix}.
  \]
  Das heißt, die Matrix ist symmetrisch und besitzt vollen Rang, weshalb sie insbesondere immer 
  invertierbar ist.
\end{itemize}
\end{remark}

\begin{proposition}[Inverse DFT]
Für ein $ n \in \N $ ist die Inverse DFT zu $ V_{n} $ gegeben durch
\[
    V_{n}^{-1}
  = \frac{1}{n} \left( e^{2\pi i j k / n} : j \in \Z_{n}, k \in Z_{m} \right)
  = \frac{1}{n} \left( \omega^{-jk} : j,k \in \Z_{n} \right).
\]
\end{proposition}

\begin{example}
Für $ n = 4 $ ist
\[
    V_{4}
  = \begin{pmatrix*}[r]
      1 &  1 &  1 &  1 \\
      1 &  i & -1 & -i \\
      1 & -1 &  1 & -1 \\
      1 & -i & -1 & i
    \end{pmatrix*}, \qquad
    V_{4}^{-1}
  = \frac{1}{4} \begin{pmatrix*}[r]
      1 &  1 &  1 &  1 \\
      1 & -i &  1 &  i \\
      1 &  1 & -1 &  1 \\
      1 &  i &  1 & -i
    \end{pmatrix*},
\]
das heißt
\[
      V_{4} \ V_{4}^{-1}
    = \frac{1}{4} \begin{pmatrix}
        4 & 0 & 0 & 0 \\
        0 & 4 & 0 & 0 \\
        0 & 0 & 4 & 0 \\
        0 & 0 & 0 & 4
      \end{pmatrix}
    = \begin{pmatrix}
        1 & 0 & 0 & 0 \\
        0 & 1 & 0 & 0 \\
        0 & 0 & 1 & 0 \\
        0 & 0 & 0 & 1
      \end{pmatrix}.
\]
\end{example}