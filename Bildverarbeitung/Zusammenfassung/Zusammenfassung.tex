\documentclass{scrartcl}

\usepackage[utf8]{inputenc}
\usepackage[ngerman]{babel}
\usepackage{microtype}

\usepackage{scrhack} % Verschiedene Hacks für die Kompatibilität anderer Pakete mit KOMA-Script.

\usepackage{libertine}
\usepackage{amsmath,amssymb,amsthm}
\usepackage[libertine]{newtxmath} % Libertine Math.
\usepackage{commath}

\usepackage{wasysym}

\usepackage[T1]{fontenc}

% Schusterjungen und Hurenkinder
\widowpenalties 3 10000 10000 100
\clubpenalties 3 10000 10000 100
\displaywidowpenalty = 10000

\usepackage[german=guillemets]{csquotes}
\usepackage{booktabs}

\usepackage{xcolor}
\usepackage[pdftex]{graphicx}
\usepackage{hyperref}

\allowdisplaybreaks % Allows pagebreaking of aligned equations

\newtheoremstyle{myplain}% name
  {\topsep}% Space above
  {\topsep}% Space below
  {\itshape}% Body font
  {}% Indent amount
  {\bfseries\sffamily}% Theorem head font
  {}%Punctuation after theorem head
  {.5em}%Space after theorem head
  {}% theorem head spec

\newtheoremstyle{mydefinition}% name
  {\topsep}% Space above
  {\topsep}% Space below
  {\normalfont}% Body font
  {}% Indent amount
  {\bfseries\sffamily}% Theorem head font
  {}%Punctuation after theorem head
  {.5em}%Space after theorem head
  {}% theorem head spec

\newtheoremstyle{myremark}% name
  {\topsep}% Space above
  {\topsep}% Space below
  {\normalfont}% Body font
  {}% Indent amount
  {\itshape\sffamily}% Theorem head font
  {}%Punctuation after theorem head
  {.5em}%Space after theorem head
  {}% theorem head spec

\theoremstyle{myplain}
\newtheorem{proposition}{Satz}[section]

\theoremstyle{mydefinition}
\newtheorem{definition}[proposition]{Definition}
\newtheorem{example}[proposition]{Beispiel}
\newtheorem{remark}[proposition]{Bemerkung}

\newcommand{\N}{\mathbb{N}}
\newcommand{\R}{\mathbb{R}}
\newcommand{\Z}{\mathbb{Z}}
\newcommand{\C}{\mathbb{C}}
\newcommand{\T}{\mathbb{T}}

\newcommand{\TODO}[1]{{\color{red}\textbf{\textsf{TODO:}}~#1}}
\usepackage[pdftex]{graphicx}  
\begin{document}

\title{Bildverarbeitung}
\subtitle{Zusammenfassung}
\author{Manuel Pauli\and{}Sebastian Schweikl}

\maketitle

\section{Mathematische Grundlagen}

\begin{definition}[Torus]
Als \emph{Torus} bezeichnet man die Menge der Äquivalenzklassen definiert durch
\[
  \T \coloneqq \R / 2\pi\Z,
\]

gesprochen \enquote{$ \R $ modulo $ 2\pi\Z $}.
\end{definition}

\begin{remark}[Torus]
Einfach gesagt: Alle reellen Zahlen, die beim Teilen durch Vielfache von $ 2\pi $ den selben Rest 
lassen, sind äquivalent, und der Torus enthält alle möglichen Reste, die dabei auftreten können. 
Beispiel:
\[
0 / 2\pi = 0 \text{ Rest } 0, \quad 
2\pi / 2\pi = 1 \text{ Rest } 0, \quad 
4\pi / 2\pi = 2 \text{ Rest } 0, \dots
\]
D.h., die Zahlen $ 0 $, $ 2\pi $ und $ 4\pi $ sind zueinander äquivalent, da sie alle den Rest $ 0 $
lassen. Man sagt, sie befinden sich in einer gemeinsamen \emph{Äquivalenzklasse}. Da es unendlich
viele Zahlen gibt, die beim Teilen durch $ 2\pi $ den Rest $ 0 $ lassen, befinden sich unendlich
viele Zahlen in dieser Äquivalenzklasse. Daher sucht man sich einen Stellvertreter (einen sog.
\emph{Repräsentanten}), um vernünftig arbeiten zu können. Es bietet sich die einfachste Zahl in der
Äquivalenzklasse an, in diesem Fall die $ 0 $, und man schreibt dann häufig $ [0] $, wenn alle 
Zahlen gemeint sind, die zur $ 0 $ äquivalent sind.

Nun wird dem einen oder anderen schon aufgefallen sein, dass der Torus auch unendlich viele 
Äquivalenzklassen besitzt (da es ja unendlich viele mögliche Reste beim Teilen gibt).
Z.B.\ ist jede Zahl aus dem Intervall $ [0, 2\pi) $ ein Repräsentant genau einer Äquivalenzklasse,
und zu jeder Äquivalenzklasse kann man einen Repräsentanten in $ [0, 2\pi) $ finden. Daher sagt man
$ \T $ ist \emph{isomorph} zu $ [0, 2\pi) $, in Zeichen
\[
  \T \simeq [0, 2\pi).
\]
Wichtig: Das heißt \emph{nicht}, dass $ \T $ das Gleiche ist wie $ [0, 2\pi) $! $ [0, 2\pi) $
ist immer noch ein Intervall, und die Elemente aus $ \T $ können zwar mit denen aus $ [0, 2\pi) $ 
identifiziert werden, aber $ \T $ hat \emph{mehr Struktur} als $ [0, 2\pi) $. Denn: Wenn ich z.B. 
die Zahl $ \pi $ aus dem Torus mit sich selber addiere, dann bekomme ich $ 0 $, denn
$ \pi + \pi = 2\pi $ und $ 2\pi / 2\pi = 1 $ mit Rest $ 0 \in \T $. Das Ergebnis ist wieder ein 
Element aus dem Torus! Wenn ich aber $ \pi \in [0, 2\pi) $ betrachte, und die selbe Rechnung
wiederhole, dann bekomme ich immer noch $ 2\pi $. Aber im Unterschied zu vorher gilt jetzt
$ 2\pi \notin [0, 2\pi) $!

Übrigens gibt es unendlich viele Intervalle, die isomorph zu $ \T $ sind, z.B.
\[
  [-\pi, \pi), (-2\pi, 0], [-2\pi, 0), [42.5\pi, 44.5\pi), \dots
\]
Jedes Intervall der Länge $ 2\pi $ ist isomorph zu $ \T $! Aber man sucht sich natürlich nur die
hübschen Intervalle raus, und das sind im wesentlichen eh nur $ [0, 2\pi) $ und $ [-\pi, \pi) $.
\end{remark}

\begin{definition}[Funktionenräume] \leavevmode
\begin{enumerate}
\item Wir bezeichnen mit $ L(\R) $ die Gesamtheit aller reellwertigen Funktionen, d.h.\ die Menge 
  aller Funktionen, die von $ \R $ von $ \R $ abbilden:
  \[
    L(\R) = \{ f \colon \R \rightarrow \R \}.
  \]
  Analog ist $ l(\Z) $ die Menge aller reellwertigen Folgen, also Funktionen, die von $ \Z $ nach
  $ \R $ abbilden:
  \[
    l(\Z) = \{ c \colon \Z \rightarrow \R \}.
  \]
  Wichtig: Die Indexmenge der Folge kommt aus $ \Z $, das Bild einer Folge das aber 
  selbstverständlich weiterhin eine Teilmenge von $ \R $ sein. Wir könnten also auch schreiben:
  \[
    l(\Z) = \{ (c_{n})_{n \in \Z} \subseteq \R \}.
  \]
\item Die Menge aller summierbaren Funktionen $ L_{1}(\R) $ ist definiert durch
\[ 
  L_{1}(\R) \coloneqq \left\{
    f \in L(\R) \colon \norm{f}_{1} \coloneqq \int_{\R} |f(t)| \dif t < \infty
  \right\}
\]
und die Menge aller summierbaren Folgen durch
\[ 
  l_{1}(\Z) \coloneqq \left\{
    c \in l(\Z) \colon \norm{f}_{1} \coloneqq \sum_{k \in \Z} |c(k)| < \infty
  \right\}
\]
Bildlich Dargestellt kann man sich dies so vorstellen:
\begin{figure}[h]
\centering
\includegraphics[width=0.5\linewidth]{Bilder/L1}
\caption{}
\label{fig:L1}
\end{figure}




\item Analog wird die Menge der quadratsummierbaren Funktionen $ L_{2}(\R) $ definiert durch
\[ 
  L_{2}(\R) \coloneqq \left\{
    f \in L(\R) \colon \norm{f}_{2} \coloneqq \sqrt{ \int_{\R} |f(t)|^{2} \dif t } < \infty
  \right\}
\]
und die Menge der quadratsummierbaren Folgen durch
\[ 
  l_{2}(\Z) \coloneqq \left\{
    c \in l(\Z) \colon \norm{c}_{2} \coloneqq \sqrt{ \sum_{k \in \Z} |c(k)|^{2} } < \infty.
  \right\}
\]
$ \norm{\bullet}_{2} $ bezeichnet man auch als die \enquote{Energie-Norm}.\\
Bildlich Dargestellt kann man sich dies so vorstellen:
\begin{figure}[h]
	\centering
	\includegraphics[width=0.5\linewidth]{Bilder/L2}
	\caption{}
	\label{fig:L1}
\end{figure}
\item Die Menge aller beschränkten Funktionen und Folgen definiert durch
\[
  L_{\infty}(\R) \coloneqq \left\{
    f \in L(\R) \colon \norm{f}_{\infty} \coloneqq
      \sup_{t \in \R} |f(t)| < \infty
  \right\}
\]
bzw.
\[
  l_{\infty}(\Z) \coloneqq \left\{
    c \in l(\Z) \colon \norm{c}_{\infty} \coloneqq
      \sup_{k \in \Z} |c(k)| < \infty
  \right\}.
\]
\item Einer geht noch: Die Menge der Funktionen und Folgen mit \emph{endlichem Träger}, geschrieben
als $ L_{00}(\R) $ bzw.\ $ l_{00}(\Z) $. Was ist mit \enquote{endlichem Träger} gemeint? Das 
bedeutet, dass der Bereich, auf dem die Funktion bzw.\ Folge lebt, nicht unendlich groß sein darf.
Formal: Es existiert ein $ N \in \N $, sodass
\[
  \left.
  \begin{array}{r}
    \{ t \in \R \colon f(t) \neq 0 \} \\
    \{ k \in \Z \colon c(k) \neq 0 \}
  \end{array}
  \right\}
  \subseteq [-N, N].
\]
\end{enumerate}
\end{definition}

\begin{remark}[$ L_{1}(\R) $-Funktionen und $ l_{1}(\Z) $-Folgen]
Wir betrachten nur Funktionen (bzw. Folgen, aber das werde ich jetzt nicht mehr dazu 
sagen), die \enquote{brav} sind. Damit ist gemeint, dass die Funktionen ein hinreichend schnelles 
Abklingverhalten gegen $ 0 $ besitzen müssen. Anschaulich gesprochen bewirkt der Betrag ja, dass 
wir einfach alles, was von der Funktion unterhalb der $ x $-Achse liegt, nach oben 
\enquote{umklappen}, sodass es nun positiv ist. Und wenn wir jetzt darüber integrieren, darf nur 
was Endliches dabei herauskommen. Dies ist eine hinreichende Forderung, damit wir die 
Fourier-Transformation zu einer Funktion überhaupt vernünftig definieren können.

Wir stellen fest, dass in $ l_{1}(\Z) $ \emph{nur} Nullfolgen (Folgen, deren
Grenzwert $ 0 $ ist) zu finden sind, z.B.\ $ (1 / k^2)_{k \in \Z} $. Achtung! Die Folge 
$ (1 / k)_{k \in \Z} $ ist zwar auch eine Nullfolge, aber die Reihe dazu konvergiert nicht 
absolut (wir haben es hier ja mit der harmonischen Reihe zu tun), und ist daher ist die Folge auch 
nicht in $ l_{1}(\Z) $.
\end{remark}

\begin{remark}[$ L_{2}(\R) $-Funktionen und $ l_{2}(\Z) $-Folgen]
Unterschied zu den \enquote{normalen} summierbaren Funktionen ist, dass hier die Zwei-Norm der
Funktion kleiner als unendlich sein muss anstatt der Eins-Norm (daher kommt ja auch der Name
\enquote{quadratsummeribar}). Leider gibt es hierfür keine so schöne geometrische Anschauung, da wir
über ganz $ \R $ integrieren, aber man kann versuchen, sich das Ganze so vorzustellen: Durch das
Quadrieren des Betrags erhält man sozusagen eine Fläche, und durch das Integrieren erzeugen wir
ein Volumen, welches dann so wie ein Schlauch an der $ x $-Achse entlang wabert. Durch das 
Wurzelziehen brechen wir das Volumen wieder herunter auf eine Fläche. Und hier endet leider schon
die Analogie. Auf dem Torus würde man jetzt noch durch $ 2\pi $ teilen (weil das die Länge eines
Intervalls ist, welches isomorph zum Torus ist), und man könnte sich die Zwei-Norm vorstellen als
Mittelwert der Fläche. Da wir aber über ganz $ \R $ integrieren und wir schlecht durch $ \infty $
teilen können, lass ma das hier bleiben und geben uns mit dem zufrieden, was wir schon haben. 

Das führt uns unweigerlich zu einer wichtigen Frage: Warum sollte man also überhaupt den Raum $ 
L_{2}(\R) $ definieren wollen? Die Antwort ist: Weil Mathematiker es immer cool finden, 
irgendwelche abgefahrenen Konzepte zu verallgemeinern. Außerdem kann man in $ L_{2}(\R) $ ein 
Skalarprodukt von Funktionen definieren, mit dem sich recht schön rechnen lässt, was eben in 
$ L_{1}(\R) $ nicht geht.

Für die mathematisch Interessieren unter uns: $ L_{2}(\R) $ liegt dicht in $ L_{1}(\R) $, und es
gilt:
\[
  L_{1}(\R) \nsubset L_{2}(\R) \quad \text{und} \quad L_{2}(\R) \nsubset L_{1}(\R).
\]
Cool ist es aber, wenn wir Funktionen im Schnitt der beiden Funktionenräume betrachten. Für solche
Funktionen kann man nämlich wieder eine Fourier-Transformierte definieren, und man hat sogar ein
Skalarprodukt.
\end{remark}

\begin{example}[Beschränkte Folge]
Betrachten wir als Beispiel einer Folge in $ l_{\infty}(\Z) $ die Folge $ ((-1)^{n} : n \in \Z) $.
Die Folge besitzt zwei Häufungspunkte $ -1 $ und $ 1 $, zwischen denen sie immer hin- und 
herspringt. Das Supremum dieser Folge ist natürlich $ 1 $, was kleiner als $ \infty $ ist. Wäre 
diese Folge auch in $ l_{1}(\Z) $? Nein, wäre sie nicht, da sie ja nicht mal konvergiert.
\end{example}

\begin{example}[$ L_{00}(\R) $-Funktion]
Die Exponentialfunktion $ \exp $ wird niemals $ 0 $, sie hat unendlichen Träger und
deshalb keine $ L_{00}(\R) $-Funktion. Die Rechtecksfunktion $ \chi_{[-1,1]} $ hingegen ist nur
im Intervall $ [-1,1] $ ungleich $ 0 $ und daher in $ L_{00}(\R) $.
\end{example}

\begin{definition}[Dirac-Puls]
Der Dirac-Puls
\[
  \delta(k) \coloneqq \delta_{0k} \coloneqq 
  \begin{cases}
    1,& k = 0, \\ 0,& k \neq 0
  \end{cases}
\]
ist eine lustige Funktion, die nur an der Stelle $ 0 $ gleich $ 1 $ ist und sonst überall $ 0 $.
\end{definition}

\begin{definition}[Abtastoperator]
Der Abtastoperator $  S_{h} : L(\R) \rightarrow l(\R) $ mit Schrittweite $ h $ ist für eine
Funktion $ f $ definiert als
\[
  (S_{h}f)(k) \coloneqq f(hk), \quad k \in \Z.
\]
Das heißt, anstatt die Funktion für alle reellen Zahlen zu betrachten, tasten wir die Funktion
an diskret vielen Stellen ab, welche alle im Abstand $ h \in \R $ zueinander sind. Wir betrachten
die Funktion also nur an den Stellen $ 0, h, -h, 2h, -2h, 3h, -3h \dots $.
\end{definition}

\section{Fourier}

\begin{definition}[Fourier-Transformation]
Für Funktionen $ f \in L_{1}(\R) $ definieren wir mit
\[
  \widehat{f} : \R \rightarrow \C, \qquad
  \widehat{f}(\xi) \coloneqq f^{\wedge}(\xi) \coloneqq 
  \int_{\R} f(t) e^{-i\xi t} \dif t, \quad \xi \in \R
\]
die Fourier-Transformierte von $ f $ und für Folgen $ c \in l_{1}(\Z) $
\[
  \widehat{c} : \Z \rightarrow \C, \qquad
  \widehat{c}(\xi) \coloneqq c^{\wedge}(\xi) \coloneqq 
  \sum_{k \in \Z} c(k) e^{-i\xi t}, \quad \xi \in \R
\]
die Fourier-Transformierte von $ c $.
\end{definition}

\begin{remark}[Fourier-Transformation]\leavevmode
\begin{itemize}
\item Was machen wir hier eigentlich? Schreiben wir einfach mal $ \widehat{f} $ als
\begin{align*}
   \int_{\R} f(t) e^{-i\xi t} \dif t
&= \int_{\R} f(t) (\cos(\xi t) - i\sin(\xi t)) \dif t \\
&= \int_{\R} f(t) \cos(\xi t) - i \int_{\R} f(t) \sin(\xi t) \dif t
\end{align*}
dann sehen wir, dass wir lediglich versuchen, $ f $ auszudrücken als Kombination von sinus- und
cosinus-Termen. Wir schauen einfach, wo $ f $ und der $ \sin $ bzw. $ \cos $ eine große Ähnlichkeit
zueinander haben (an der Stelle wird das Integral dann groß) und finden so heraus, welchen 
\enquote{Anteil} die Frequenz $ \xi $ am Signal $ f $ hat. Dass wir hier die doofe imaginäre 
Einheit $ i $ mit drin haben, liegt halt einfach daran, dass wir die Identität
\[
  e^{ix} = \cos(x) + i \sin(x)
\]
ausgenutzt haben, um die Fouriertransformation besonders elegant zu schreiben. Man hätte auch für
Real- und Imaginärteil zwei gesonderte Fouriertransformationen definieren können. Aber das soll uns
hier nicht weiter stören. Außerdem kann man halt mit einer Exponentialfunktion schöner rechnen 
(z.B. ist die Stammfunktion der Exponentialfunktion wieder die Exponentialfunktion). Das ist 
eigentlich alles, was dahinter steckt. Will man die imaginäre Einheit ganz wegbekommen, geht man 
im diskreten Fall einfach über zur \emph{Diskreten Cosinus-Transformation}.
\item Wichtig: Mit der Fouriertransformation finden wir zwar heraus, welche Frequenzen im Singal 
stecken,
aber wir wissen nicht, an welcher Stelle bzw. zu welchem Zeitpunkt die entsprechende Frequenz 
auftritt! Wir haben keine Lokalität, da wir ja über ganz $ \R $ integrieren. Das ist ein wichtiger 
Unterschied zur \emph{Gabor-Transformation}, wo wir unser einer Fensterfunktion bedienen, um so 
Frequenzen besser lokalisieren zu können $ \smiley $.
\item Warum brachen wir $ L_{1}(\R) $-Funktionen? Wie vorher schon erwähnt, ist das eine 
hinreichende Bedingung, dass $ \widehat{f} $ überhaupt existiert:
\[
  \widehat{f} \leq |\widehat{f}| = \left| \int_{\R} f(t) e^{-i\xi t} \dif t \right|
  \leq \int_{\R} |f(t)| \underbrace{|e^{-i\xi t}|}_{= 1} \dif t = \int_{\R} |f(t)| \dif t < \infty.
\]
Das Argument lässt sich analog auf $ l_{1}(\Z) $-Folgen übertragen.
\end{itemize}
\end{remark}

\begin{definition}[Translations- und Skalierungsoperator] \leavevmode
\begin{enumerate}
\item Der Translationsoperator $ \tau_{y} $ mit $ y \in \R $ angewendet auf eine Funktion $ f $ ist 
definiert als
\[ \tau_{y} \ f \coloneqq f(\bullet + y). \]
\begin{figure}[h]
	\centering
	\includegraphics[width=0.5\linewidth]{Bilder/translation}
	\caption{}
	\label{fig:L1}
\end{figure}
\item Der Skalierungsoperator $ \sigma_{h} $ mit $ h \in \R \setminus \{ 0 \} $ angewendet auf eine 
Funktion $ f $ ist definiert als
\[ \sigma_{h} \ f \coloneqq f(h \cdot \bullet). \]
\begin{figure}[h]
	\centering
	\includegraphics[width=0.5\linewidth]{Bilder/Skalierung}
	\caption{}
	\label{fig:L1}
\end{figure}
\end{enumerate}
\end{definition}

\begin{remark}[Translations- und Skalierungsoperator] \leavevmode
\begin{enumerate}
\item der Translationsoperator verschiebt eine Funktion auf der $ x $-Achse um $ y $ Einheiten nach 
links oder rechts:
\par
\begin{center}
  \begin{tabular}{rl} \toprule
  Wert von $ y $ & Effekt auf $ f $ \\ \midrule
  $ y > 0 $ & Verschiebung nach \emph{links} \\
  $ y = 0 $ & Keine Verschiebung \\
  $ y < 0 $ & Verschiebung nach \emph{rechts} \\ \bottomrule
  \end{tabular}
\end{center}
\item Der Skalierungsoperator streckt oder staucht eine Funktion um den Faktor $ h $ und kann sie 
sogar an der $ y $-Achse spiegeln:\par
\begin{center}
  \begin{tabular}{rl} \toprule
  Wert von $ h $ & Effekt auf $ f $ \\ \midrule
  $ h > 1 $ & Stachung \\
  $ h = 1 $ & Kein Effekt \\
  $ 0 < h < 1 $ & Streckung \\ \midrule
  $ h = 0 $ & Um Gottes Willen! Das ist pfui-gack. \\ \midrule
  $ -1 < h < 0 $ & Streckung und Spiegelung an der $ y $-Achse \\
  $ h = -1 $ & Nur Spiegelung an der $ y $-Achse \\
  $ h < -1 $ & Stauchung und Spiegelung an der $ y $-Achse \\ \bottomrule
  \end{tabular}
\end{center}
\item Translation und Skalierung sind invertierbar, d.h.\ man kann ihre Auswirkungen wieder
rückgängig machen:
\[
  \tau_{y} \ (\tau_{-y} \ f) = \tau_{-y} \ (\tau_{y} \ f) = f \quad \text{und} \quad
  \sigma_{h} \ (\sigma_{1/h} \ f) = \sigma_{1/h} \ (\sigma_{h} \ f) = f.
\]
\end{enumerate}
\end{remark}

\begin{definition}[Faltung]
Seien $ f, g \in L(\R) $ und $ c,d \in l(\Z) $. Dann ist die Faltung zweier Funktionen definiert als
\[
  f * g \coloneqq \int_{\R} f(\bullet - t) \cdot g(t) \dif t \in L(\R)
\]
und die Faltung zweier Folgen als
\[
  c * d \coloneqq \sum_{k \in \Z} c(\bullet - k) \cdot d(k) \in l(\Z).
\]
Die Faltung einer Funktion $ f $ mit einer Folge $ c $ ist definiert durch
\[
  c * f \coloneqq f * c \coloneqq \sum_{k \in \Z} f(\bullet - k) \cdot d(k) \in L(\R).
\]
\end{definition}

\begin{remark}[Eigenschaften der Faltung]
Aus der Linearität des Integrals und den Gruppenoperationen auf $ \R $ lassen sich folgende 
Eigenschaften der Faltung für Funktionen oder Folgen $ f, g, h $ herleiten:
\begin{itemize}
\item Kommutativität: $ f * g = g * f $
\item Assoziativität: $ (f * g) * h = f * (g * h) $
\item Distributivität: $ f * (g + h) = f * g + f * h $
\item Skalare Multiplikation: $ a \cdot (f * g) = (a \cdot f) * g = f * (a \cdot g) $, wobei
  $ a \in \C $ eine beliebige Konstante ist.
\end{itemize}
\end{remark}
\begin{remark}[Interpretation der Faltung]
Die Faltung kann aufgefasst werden als Produkt zweier Funktionen oder Folgen, welches wieder eine
Funktion bzw. Folge liefert. Wie kann man sich die Faltung geometrisch vorstellen? Betrachten wir 
als Beispiel zwei Funktionen $ f $ und $ g $ und die Faltung $ f * g $. Was dabei passiert, ist 
Folgendes: Wir halten die Funktion $ g $ fest und lassen $ f $ einmal komplett von ganz links nach 
ganz rechts über die $ x $-Achse wandern. Dort, wo sich $ f $ und $ g $ überlagern und eine große 
\enquote{Gemeinsamkeit} miteinander haben, wird auch das Integral groß. Dort, wo beide Funktionen 
keine große Gemeinsamkeit miteinander haben, wird das Integral klein. Die Faltung ist also eine 
Methode, um feststellen zu können, wie \emph{lokal} ähnlich (nicht global!) sich zwei Funktionen 
sind.

Je nachdem, welche Funktion man für $ g $ wählt, lassen sich mit der Faltung unterschiedliche 
interessante andere Funktionen erzeugen. Wikipedia meint, dass eine Faltung $ f * g $ einen
\enquote{gewichteten Mittelwert} von $ f $ darstellt, wobei die Gewichtung durch $ g $ vorgegeben 
ist. Diese Argumentation versteht man eigentlich erst, wenn man sich \emph{zyklische Faltungen}
anschaut, wo nicht über ganz $ \R $ integriert wird, sondern über ein Kompaktum. Dann wird das
Integral nämlich noch durch die Länge des Kompaktums dividiert, sodass man tatsächlich eine Art
Durschnitt hat. Und hey, das kommt uns doch jetzt irgendwie von der Definition der $ L_{2}(\R) $-
Funktionen bekannt vor!
\end{remark}

\begin{example}[Faltung]
Falten wir doch einmal die Rechtecksfunktion $ \chi_{[-1,1]} $ mit sich selbst. Man definiert
\[
  \chi_{[-1,1]}(x) = \begin{cases} 1, & x \in [-1,1] \\ 0, & \text{sonst}. \end{cases}
\]
Dann ist
\begin{align*}
   (\chi_{[-1,1]} * \chi_{[-1,1]})(x) 
&= \int_{\R} \chi_{[-1,1]}(x - t) \cdot \chi_{[-1,1]}(t) \dif t
 = \int_{\R} \chi_{[x - 1, x + 1]}(t) \cdot \chi_{[-1,1]}(t) \dif t \\
&= \int_{\R} \chi_{[x - 1, x + 1] \cap [-1,1]}(t) \dif t
 = \int_{[x - 1, x + 1] \cap [-1,1]} 1 \dif t \\
&= \begin{cases}
      \int_{-1}^{x+1} 1 \dif t = 2 + x, & -2 \leq x \leq 0, \\
      \int_{x-1}^{1} 1 \dif t = 2 - x, & 0 < x \leq 2, \\
      0, & \text{sonst},
   \end{cases} \\
&\eqcolon \Delta_{[-2,2]}(x).
\end{align*}
Man erhält also die Dreiecksfunktion auf dem Intervall $ [-2,2] $. Was bedeutet das? Naja, die
$ \Delta_{[-2,2]}(x) $ an der Stelle $ x $ gibt genau die Fläche an, die zwischen den beiden 
Rechtecksfunktionen gerade eingeschlossen wird, wenn man eine Rechtecksfunktion um $ x $ Einheiten
verschiebt. Im Fall von $ x = 0 $ überlappen sich beide Rechtecksfunktionen ganz genau, und deren
Flächeninhalt ist $ 2 $. Dies ist genau der Wert von $ \Delta_{[-2,2]}(2) $! Verschiebt man eine
Rechtecksfunktion um $ 1 $ Einheit nach links oder rechts, dann wird nur noch die Hälfte der
Fläche zwischen beiden Rechtecksfunktionen eingeschlossen, also Flächeninhalt $ 1 $. Und genau das
kommt bei $ \Delta_{[-2,2]}(x) $ heraus, wenn man $ x = 1 $ oder $ x = -1 $ einsetzt.

Die geometrische Interpretation einer Faltung ist also sehr vielfältig und hängt sehr stark von den
beiden Funktionen ab, die miteinander gefaltet werden.\\
Bildlich Dargestellt kann man sich dies so vorstellen:
\begin{figure}[h]
	\centering
	\includegraphics[width=0.7\linewidth]{Bilder/Faltung}
	\caption{}
	\label{fig:L1}
\end{figure}
\end{example}
\begin{remark}[Eigenschaften der Fouriertransformation und deren graphische Deutung]:
	 %%Hier könnten wir zu jeder Eigenschaft kurz schreiben was sie bedeuten und falls nötig oder möglich durch Bilder
	%%veranschaulichen
	\begin{enumerate}
		\item Translation für $ f \in L_1(\R)  $ und $ y \in \R $ 
		\[ \widehat{\tau_{y} \ f}(\xi) = e^{iy\xi}\widehat{f}(\xi), \xi \in \R \]
		Wie bereits oben erwähnt verschiebt eine Translation eine Funktion um den Wert $ y $. Verschiebt man nun die ursprüngliche Funktion des Signal, sollte sich an den Anteilen der Frequenzen $ \xi $ nichts ändern. Wo kommt aber nun dieser seltsame Faktor $ e^{iy\xi} $ her? Dazu müssen wir uns wieder in Erinnerung rufen, dass sich die Fouriertransformation auch so schreiben lässt: $ \int_{\R} f(t) \cos(\xi t) - i \int_{\R} f(t) \sin(\xi t) \dif t $\\
		$ \rightarrow $ Es gibt einen Imaginärteil und einen Realteil. Durch die Translation ändert sich nur die Aufteilung zwischen diesen zwei Teilen. Der Gesamtanteil bleibt gleich. Diese Umverteilung geschieht durch den Vorfaktor $ e^{iy\xi} $  (Bild einfügen)
		\item Skalierung
		\item Faltung
		\item Ableitungseigenschaften
		\item Inverse Fouriertransformation
	\end{enumerate}
\end{remark}
\end{document}
