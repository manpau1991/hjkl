\documentclass[oneside]{scrbook}

\usepackage[utf8]{inputenc}
\usepackage[ngerman]{babel}
\usepackage{microtype}

\usepackage{scrhack} % Verschiedene Hacks für die Kompatibilität anderer Pakete mit KOMA-Script.

\usepackage{libertine}
\usepackage{amsmath,amssymb,amsthm}
\usepackage[libertine]{newtxmath} % Libertine Math.
\usepackage{commath}

\usepackage{wasysym}

\usepackage[T1]{fontenc}

% Schusterjungen und Hurenkinder
\widowpenalties 3 10000 10000 100
\clubpenalties 3 10000 10000 100
\displaywidowpenalty = 10000

\usepackage[german=guillemets]{csquotes}
\usepackage{booktabs}

\usepackage{xcolor}
\usepackage[pdftex]{graphicx}
\usepackage{tikz}
\usetikzlibrary{shapes}
\usepackage{hyperref}

\allowdisplaybreaks % Allows pagebreaking of aligned equations

\newtheoremstyle{myplain}% name
  {\topsep}% Space above
  {\topsep}% Space below
  {\itshape}% Body font
  {}% Indent amount
  {\bfseries\sffamily}% Theorem head font
  {}%Punctuation after theorem head
  {.5em}%Space after theorem head
  {}% theorem head spec

\newtheoremstyle{mydefinition}% name
  {\topsep}% Space above
  {\topsep}% Space below
  {\normalfont}% Body font
  {}% Indent amount
  {\bfseries\sffamily}% Theorem head font
  {}%Punctuation after theorem head
  {.5em}%Space after theorem head
  {}% theorem head spec

\newtheoremstyle{myremark}% name
  {\topsep}% Space above
  {\topsep}% Space below
  {\normalfont}% Body font
  {}% Indent amount
  {\itshape\sffamily}% Theorem head font
  {}%Punctuation after theorem head
  {.5em}%Space after theorem head
  {}% theorem head spec

\theoremstyle{myplain}
\newtheorem{proposition}{Satz}[chapter]

\theoremstyle{mydefinition}
\newtheorem{definition}[proposition]{Definition}
\newtheorem{example}[proposition]{Beispiel}
\newtheorem{remark}[proposition]{Bemerkung}

\newcommand{\N}{\mathbb{N}}
\newcommand{\R}{\mathbb{R}}
\newcommand{\Z}{\mathbb{Z}}
\newcommand{\C}{\mathbb{C}}
\newcommand{\T}{\mathbb{T}}

\DeclareMathOperator{\sinc}{sinc}

\newcommand{\TODO}[1]{{\color{red}\textbf{\textsf{TODO:}}~#1}}

\begin{document}

\title{Bildverarbeitung}
\subtitle{Zusammenfassung}
\author{Manuel Pauli\and{}Sebastian Schweikl}

\maketitle

\section{Mathematische Grundlagen}

\begin{definition}[Funktionenräume] \leavevmode
\begin{itemize}
\item Wir bezeichnen mit $ L(\R) $ die Gesamtheit aller reellwertigen Funktionen, d.h.\ die Menge 
  aller Funktionen, die von $ \R $ von $ \R $ abbilden:
  \[
    L(\R) = \{ f : \R \rightarrow \R \}.
  \]
  Analog ist $ l(\Z) $ die Menge aller reellwertigen Folgen, also Funktionen, die von $ \Z $ nach
  $ \R $ abbilden:
  \[
    l(\Z) = \{ c : \Z \rightarrow \R \}.
  \]
  Wichtig: Die Indexmenge der Folge kommt aus $ \Z $, das Bild einer Folge das aber 
  selbstverständlich weiterhin eine Teilmenge von $ \R $ sein. Wir könnten also auch schreiben:
  \[
    l(\Z) = \{ (c_{n})_{n \in \Z} \subseteq \R \}.
  \]
\item Die Menge aller summierbaren Funktionen $ L_{1}(\R) $ ist definiert durch
\[ 
  L_{1}(\R) \coloneqq \left\{
    f \in L(\R) : \norm{f}_{1} \coloneqq \int_{\R} |f(t)| \dif t < \infty
  \right\}
\]
und die Menge aller summierbaren Folgen durch
\[ 
  l_{1}(\Z) \coloneqq \left\{
    c \in l(\Z) : \norm{f}_{1} \coloneqq \sum_{k \in \Z} |c(k)| < \infty
  \right\}
\]
D.h., wir betrachten hier nur Funktionen (bzw. Folgen, aber das werde ich jetzt nicht mehr dazu 
sagen), die \enquote{brav} sind. Damit ist gemeint, dass die
Funktionen ein hinreichend schnelles Abklingverhalten gegen $ 0 $ besitzen müssen. Anschaulich 
gesprochen bewirkt der Betrag ja, dass wir einfach alles, was von der Funktion unterhalb der
$ x $-Achse liegt, nach oben \enquote{umklappen}, sodass es nun positiv ist. Und wenn wir jetzt 
darüber integrieren, darf nur was Endliches dabei herauskommen. Dies ist eine hinreichende 
Forderung, damit wir die Fourier-Transformation zu einer Funktion überhaupt vernünftig definieren 
können.

Wir stellen fest, dass in $ l_{1}(\Z) $ \emph{nur} absolut konvergente Nullfolgen (Folgen, deren
Grenzwert $ 0 $ ist) zu finden sind.
\item Analog wird die Menge der quadratsummierbaren Funktionen $ L_{2}(\R) $ definiert durch
\[ 
  L_{2}(\R) \coloneqq \left\{
    f \in L(\R) : \norm{f}_{2} \coloneqq \sqrt{ \int_{\R} |f(t)|^{2} \dif t } < \infty
  \right\}
\]
Unterschied zu den \enquote{normalen} summierbaren Funktionen ist, dass hier die Zwei-Norm der
Funktion kleiner als unendlich sein muss anstatt der Eins-Norm (daher kommt ja auch der Name
\enquote{quadratsummeribar}). Leider gibt es hierfür keine so schöne geometrische Anschauung, da wir
über ganz $ \R $ integrieren, aber man kann versuchen, sich das Ganze so vorzustellen: Durch das
Quadrieren des Betrags erhält man sozusagen eine Fläche, und durch das Integrieren erzeugen wir
ein Volumen, welches dann so wie ein Schlauch an der $ x $-Achse entlang wabert. Durch das 
Wurzelziehen brechen wir das Volumen wieder herunter auf eine Fläche. Und hier endet leider schon
die Analogie. Auf dem Torus würde man jetzt noch durch $ 2\pi $ teilen (weil das die Länge eines
Intervalls ist, welches isomorph zum Torus ist), und man könnte sich die Zwei-Norm vorstellen als
Mittelwert der Fläche. Da wir aber über ganz $ \R $ integrieren und wir schlecht durch $ \infty $
teilen können, lass ma das hier bleiben und geben uns mit dem zufrieden, was wir schon haben. 

Das führt uns unweigerlich zu einer wichtigen Frage: Warum sollte man also überhaupt den Raum $ 
L_{2}(\R) $ definieren wollen? Die Antwort ist: Weil Mathematiker es immer cool finden, 
irgendwelche abgefahrenen Konzepte zu verallgemeinern. Außerdem kann man in $ L_{2}(\R) $ ein 
Skalarprodukt von Funktionen definieren, mit dem sich recht schön rechnen lässt, was eben in 
$ L_{1}(\R) $ nicht geht.

Für die mathematisch Interessieren unter uns: $ L_{2}(\R) $ liegt dicht in $ L_{1}(\R) $, und es
gilt:
\[
  L_{1}(\R) \nsubset L_{2}(\R) \quad \text{und} \quad L_{2}(\R) \nsubset L_{1}(\R).
\]
Cool ist es aber, wenn wir Funktionen im Schnitt der beiden Funktionenräume betrachten. Für solche
Funktionen kann man nämlich wieder eine Fourier-Transformierte definieren, und man hat sogar ein
Skalarprodukt.

\item Einer geht noch: Die Menge aller beschränkten Funktionen und Folgen definiert durch
\[
  L_{\infty}(\R) \coloneqq \left\{
    f \in L(\R) : \norm{f}_{\infty} \coloneqq
      \sup_{t \in \R} |f(t)| < \infty
  \right\}
\]
bzw.
\[
  l_{\infty}(\Z) \coloneqq \left\{
    c \in l(\Z) : \norm{c}_{\infty} \coloneqq
      \sup_{k \in \Z} |c(k)| < \infty
  \right\}.
\]
Betrachten wir als Beispiel einer Folge in $ l_{\infty}(\Z) $ die Folge $ ((-1)^{n} : n \in \Z) $.
Die Folge besitzt zwei Häufungspunkte $ -1 $ und $ 1 $, zwischen denen sie immer hin- und 
herspringt. Das Supremum dieser Folge ist natürlich $ 1 $, was kleiner als $ \infty $ ist. Wäre 
diese Folge auch in $ l_{1}(\Z) $? Nein, wäre sie nicht, da sie nicht mal konvergiert.
\end{itemize}
\end{definition}

\begin{definition}[Dirac-Puls]
Der Dirac-Puls
\[
  \delta(k) \coloneqq \delta_{0k} \coloneqq 
  \begin{cases}
    1,& k = 0, \\ 0,& k \neq 0.
  \end{cases}
\]
ist eine lustige Funktion, die nur an der Stelle $ 0 $ gleich $ 1 $ ist und sonst überall $ 0 $.
\end{definition}

\begin{definition}[Abtastoperator]
Der Abtastoperator $  S_{h} : L(\R) \rightarrow l(\R) $ mit Schrittweite $ h $ ist für eine
Funktion $ f $ definiert als
\[
  (S_{h}f)(k) \coloneqq f(hk), \quad k \in \Z.
\]
Das heißt, anstatt die Funktion für alle reellen Zahlen zu betrachten, tasten wir die Funktion
an diskret vielen Stellen ab, welche alle im Abstand $ h \in \R $ zueinander sind. Wir betrachten
die Funktion also nur an den Stellen $ 0, h, -h, 2h, -2h, 3h, -3h \dots $.
\end{definition}

\section{Fourier}

\begin{definition}[Fourier-Transformation]
Für Funktionen $ f \in L_{1}(\R) $ definieren wir mit
\[
  \widehat{f} : \R \rightarrow \C, \qquad
  \widehat{f}(\xi) \coloneqq f^{\wedge}(\xi) \coloneqq 
  \int_{\R} f(t) e^{-i\xi t} \dif t, \quad \xi \in \R
\]
die Fourier-Transformierte von $ f $.

Was machen wir hier eigentlich? Schreiben wir einfach mal $ \widehat{f} $ als
\begin{align*}
   \int_{\R} f(t) e^{-i\xi t} \dif t
&= \int_{\R} f(t) (\cos(\xi t) - i\sin(\xi t)) \dif t \\
&= \int_{\R} f(t) \cos(\xi t) - i \int_{\R} f(t) \sin(\xi t) \dif t
\end{align*}
dann sehen wir, dass wir lediglich versuchen, $ f $ auszudrücken als Kombination von sinus- und
cosinus-Termen. Wir schauen einfach, wo $ f $ und der $ \sin $ bzw. $ \cos $ eine große Ähnlichkeit
zueinander haben (an der Stelle wird das Integral dann groß) und finden so heraus, welchen Anteil
welche Frequenz am Signal $ f $ hat. Dass wir hier die doofe imaginäre Einheit $ i $ mit drin haben,
liegt halt einfach daran, dass wir die Identität
\[
  e^{ix} = \cos(x) + i \sin(x)
\]
ausgenutzt haben, um die Fouriertransformation besonders elegant zu schreiben. Man hätte auch für
Real- und Imaginärteil zwei gesonderte Fouriertransformationen definieren können. Aber das soll uns
hier nicht weiter stören. Außerdem kann man halt mit einer Exponentialfunktion schöner rechnen 
(z.B. ist die Ableitung der Exponentialfunktion wieder die Exponentialfunktion). Das ist eigentlich
alles, was dahinter steckt.

Wichtig: Mit der Fouriertransformation finden wir zwar heraus, welche Frequenzen im Singal stecken,
aber wir wissen nicht, an welcher Stelle welche Frequenz auftritt! Wir haben keine Lokalität, da wir
ja über ganz $ \R $ integrieren. Das ist ein wichtiger Unterschied zur Gabor-Transformation, wo wir
unser einer Fensterfunktion bedienen, um so Frequenzen besser lokalisieren zu können $ \smiley $.

Warum brachen wir $ L_{1}(\R) $-Funktionen? Wie vorher schon erwähnt, ist das eine 
hinreichende Bedingung, dass $ \widehat{f} $ überhaupt existiert:
\[
  \widehat{f} \leq |\widehat{f}| = \left| \int_{\R} f(t) e^{-i\xi t} \dif t \right|
  \leq \int_{\R} |f(t)| \underbrace{|e^{-i\xi t}|}_{= 1} \dif t = \int_{\R} |f(t)| \dif t < \infty.
\]
\end{definition}

\end{document}