\documentclass[oneside]{scrbook}

\usepackage[utf8]{inputenc}
\usepackage[ngerman]{babel}
\usepackage{microtype}

\usepackage{scrhack} % Verschiedene Hacks für die Kompatibilität anderer Pakete mit KOMA-Script.

\usepackage{libertine}
\usepackage{amsmath,amssymb,amsthm}
\usepackage[libertine]{newtxmath} % Libertine Math.
\usepackage{commath}

\usepackage{wasysym}

\usepackage[T1]{fontenc}

% Schusterjungen und Hurenkinder
\widowpenalties 3 10000 10000 100
\clubpenalties 3 10000 10000 100
\displaywidowpenalty = 10000

\usepackage[german=guillemets]{csquotes}
\usepackage{booktabs}

\usepackage{xcolor}
\usepackage[pdftex]{graphicx}
\usepackage{tikz}
\usetikzlibrary{shapes}
\usepackage{hyperref}

\allowdisplaybreaks % Allows pagebreaking of aligned equations

\newtheoremstyle{myplain}% name
  {\topsep}% Space above
  {\topsep}% Space below
  {\itshape}% Body font
  {}% Indent amount
  {\bfseries\sffamily}% Theorem head font
  {}%Punctuation after theorem head
  {.5em}%Space after theorem head
  {}% theorem head spec

\newtheoremstyle{mydefinition}% name
  {\topsep}% Space above
  {\topsep}% Space below
  {\normalfont}% Body font
  {}% Indent amount
  {\bfseries\sffamily}% Theorem head font
  {}%Punctuation after theorem head
  {.5em}%Space after theorem head
  {}% theorem head spec

\newtheoremstyle{myremark}% name
  {\topsep}% Space above
  {\topsep}% Space below
  {\normalfont}% Body font
  {}% Indent amount
  {\itshape\sffamily}% Theorem head font
  {}%Punctuation after theorem head
  {.5em}%Space after theorem head
  {}% theorem head spec

\theoremstyle{myplain}
\newtheorem{proposition}{Satz}[chapter]

\theoremstyle{mydefinition}
\newtheorem{definition}[proposition]{Definition}
\newtheorem{example}[proposition]{Beispiel}
\newtheorem{remark}[proposition]{Bemerkung}

\newcommand{\N}{\mathbb{N}}
\newcommand{\R}{\mathbb{R}}
\newcommand{\Z}{\mathbb{Z}}
\newcommand{\C}{\mathbb{C}}
\newcommand{\T}{\mathbb{T}}

\DeclareMathOperator{\sinc}{sinc}

\newcommand{\TODO}[1]{{\color{red}\textbf{\textsf{TODO:}}~#1}}
\usepackage[pdftex]{graphicx}  
\begin{document}

\newcommand{\INT}[1]{\text{d#1}}
\newcommand{\INTM}[1]{\text{d} #1}
\newcommand{\INTMM}[2]{\text{d#1}_{#2}}
\renewcommand{\T}[1]{\text{#1 }}
\newcommand{\DIFF}[2]{\frac{\INT{#1}}{\INT{#2}}}
\renewcommand{\theta}{\vartheta}

\title{Bildverarbeitung}
\subtitle{Fragen}
\author{Manuel Pauli\and{}Sebastian Schweikl\and{}Thomas Lang}

\maketitle
\section{Allgemein} 
\begin{description}
	\item[Erkläre Operationen: Skalierung, Translation]      
	\item[Erkläre das Prinzip einer Lochkamera]
      Eine Lochkamera ist der Vorläufer einer modernen Kamera. Dabei fällt
      das natürliche Licht durch ein kleines Loch (die Blende) ein. Das Bild
      trifft dann auf der Rückwand der Kamera auf. Bedingt durch die Geometrie
      steht das Bild auf dem Kopf, dessen Höhe/Breite wird dabei durch den
      Abstand des Objektes zur Blende, des Abstandes der Blende zur Rückwand
      und von der Größe des Objektes und der Blende bestimmt.
\end{description}	

\section{Fourier}
\subsection{Fouriertransformation}
\begin{description}
	\item[Nenne die Gleichung]
      Für eine Funktion $f \in L_1(\R)$ ist die Fouriertransformation durch
      $$ \hat f(\xi) = \int_\R f(t) e^{-i\xi t}\INT{t} $$
      für alle $\xi \in \R$ definiert.
    \item[Das ist die Gleichung für $L_1$-Funktionen. Wie sieht das im $L_2$ aus?]
      Die Fouriertransformation im Signalraum $L_2$ kann ganz analog zum obigen Fall verwendet werden.
      Die resultierende Fouriertransformierte liegt dann in $L_1(\R) \cap L_2(\R)$.\\
      Formal beruht dies darauf, dass der Raum $L_1(\R)\cap L_2(\R)$ dicht in $L_2(\R)$ liegt. Deshalb
      lässt sich also immer eine Funktionenfolge aus $L_1(\R)\cap L_2(\R)$ finden, die im Grenzwert
      gegen unsere gesuchte Fouriertransformierte konvergiert\footnote{Geht immer, zwecks Banachraum und so.}.
	\item[Wie sieht das Ergebnis der FT bei einer reellwertigen Funktion aus?]
    \item[Was sagt der Satz von Parseval/Plancherel?]%%copy
      Der Satz von Parseval/Plancherel beschreibt den Wechsel zwischen einem 
      Skalarprodukt im Zeit- und einem Skalarprodukt im Frequenzbereich.\\
      Formal gilt für Funktionen $f, g \in L_1(\R)$:
      $$ \int_\R f(t)g(t)\INT{t} = \frac{1}{2\pi}\int_\R\hat f(\theta)\overline{\hat g(\theta)}\INTM{\theta} $$
      Im Speziellen für $f=g$ folgt, dass
      $$ \|f\|_1^2 = \frac{1}{2\pi}\|\hat f\|_2^2 $$
      gilt, also dass hier (bis auf Normierung) eine Isometrie vorliegt.
    \item[Sie sprachen von Isometrie, warum ist dann der konstante Faktor vor dem rechten Term?]%% copy
      Dieser Faktor kommt daher, weil wir bei der Definition der Fouriertransformation keinen Faktor
      dabei hatten. Hätte man dort einen Faktor $1/\sqrt{2\pi}$ hinzugefügt, so hätte man hier eine
      perfekte Isometrie, was man ja mit Energieerhaltung\footnote{Die $2$-Norm $\|\cdot\|_2$ beschreibt ja
      i.W. Energie.} identifizieren kann.
	\item[Wie sieht die inverse Fouriertransformation aus?]
      Unter der Voraussetzung, dass sowohl für $f$ als auch für $\hat f$ gilt, dass
      diese Funktionen aus $L_1$ sind, ist die inverse Fouriertransformation 
      definiert als
      $$ f(\xi) = \left(\hat f\right)^{\wedge{}}(\xi) = \frac{1}{2\pi}\int_\R \hat f(t)e^{i\xi t}\INT{t} $$      
	\item[Zusammenhang zwischen Faltung und Fouriertransformation]
      Ein wichtiger Zusammenhang ist, dass sich die Faltung zweier Funktion als Produkt ihrer 
      Fouriertransformierten darstellen lässt (vorausgesetzt deren Existenz).\\
      Es gilt also:
      
      \begin{equation*}
        \begin{split}
          (f*c)^{\widehat{}}(\xi) &\stackrel{\mathrm{def}}= \int_\R (f*c)(t) e^{-i\xi t}\INT{t} \\
          &\stackrel{\mathrm{def}}= \int_\R \int_\R f(t-x)c(x)\INT{x} e^{-i\xi t} \INT{t} \\
          &\stackrel{\mathrm{Tonelli}}= \int_\R \int_\R f(t-x)c(x) e^{-i\xi t} \INT{x}\INT{t} \\
          &\stackrel{t=t+x}= \int_\R \int_\R f(t)c(x) e^{-i\xi (t+x)} \INT{x}\INT{t} \\
          &= \int_\R \int_\R f(t)e^{-i\xi t} c(x) e^{-i\xi x} \INT{x}\INT{t} \\
          &\stackrel{\mathrm{Tonelli}}= \int_\R f(t)e^{-i\xi t} \INT{t}\int_\R c(x) e^{-i\xi x} \INT{x} \\ 
          &= \hat f(\xi) \hat c(\xi)
        \end{split}
      \end{equation*}

	\item[Erkläre den Weg zur FFT]
      Der Weg von der DFT zur FFT ist der, dass man einen Baumalgorithmus implementiert. Konkret arbeitet
      ein Teilbaum alle Folgenglieder mit geradem Index ab, während der andere Teilbaum den Rest bearbeitet,
      wobei sich natürlich die Indexmenge in den Teilbäumen vom Vaterknoten verändert.

      Formal habe der Vaterknoten die Indexmenge $\Z_n = \{0,\ldots,n-1\}$ zur Folge $c$ und sei $n=2*m, m\in\N$.
      Dann ergibt sich für die Teilbäume, dass der eine alle Folgenglieder $c(2k)$ und der andere alle Folgenglieder
      $c(2k+1), k \in \Z_m = \{0,\ldots,m-1\}$ verarbeitet.
	\item[Wie schnell ist die FFT?]
      Da man dabei einen Baumalgorithmus implementiert, hat man also \emph{nur} $\log n$ Ebenen abzuarbeiten, wobei
      man immer noch alle Folgenglieder $c(k),k\in\Z_n$ verarbeiten muss, es entsteht also ein Aufwand 
      in $\mathcal{O}(n \log n)$. Nachzuweisen ist dies mit dem Mastertheorem.
    \item[Theorie schön und gut, aber wozu braucht man die Fouriertransformation nun in der Praxis?]%%copy
      Mit der ganzen Theorie (insbesondere mit dem Zusammenhang zwischen der Faltung und der Fouriertransformation)
      können Bildverarbeitungsalgorithmen in verschiedenster Hinsicht optimiert werden. Beispielsweise
      um die reine Berechnung zu beschleunigen (Multiplikation vs Faltung) oder um die numerische Stabilität
      zu erhöhen.

      Der Standardweg ist dabei, dass man das Signal fouriertransformiert, auf dem Ergebnis die Berechnungen
      durchführt und anschließend wieder zurücktransformiert.
	\item[Wie funktioniert das JPEG-Kompressionsverfahren mit Hilfe der DFT?]
      Die Idee ist hierbei, die Tatsache der Abhängigkeitsfreiheit zu nutzen um den Algorithmus massiv parallel
      zu machen. Dabei ist noch nicht mal die FFT gemeint, sondern das Vorgehen das Bild in Makroblöcke der
      fixen Größe $8x8$ aufzuspalten. Diese fixe Größe kommt daher, dass man also pro Makroblock 16 $1\times 4$-
      Vektoren vorliegen hat, welche z.B. auf Graphikkarten oder modernen CPUs sehr effizient verarbeitet werden
      können. Falls die Dimensionen des Bildes keine Vielfachen von $8$ sind, so können die fehlenden Ränder
      z.B. auf $0$ gesetzt oder das Bild periodisch fortgesetzt werden.

      Hat man nun die Makroblöcke vorliegen, so wendet man auf diese die Transformation an (sei es die DFT, die
      FFT oder die DCT). Das Ergebnis wird dann mit einem Kompressionsverfahren verknüpft, um die Größe jedes
      Makroblockes (und damit die Bildgröße) dramatisch zu verringern. Beispiele dafür wäre z.B. der Huffman-Code,
      welcher eindeutige Codes unterschiedlicher Länge generiert, die die ursprünglichen Werte darstellen. Dieses
      Encoding wäre sogar verlustfrei.

      Bei der Dekompression geht man entsprechend umgekehrt vor: Man Ersetzt die gekürzten Codes mit den ursprünglichen
      Werten und wendet die inverse Transformation (IDFT, IFFT, IDCT) an, um das alte Bild z.B. darstellen zu können.
    \item[Macht das JPEG wirklich so?]%%copy
      Nein, da die DFT/FFT u.U. imaginäre Werte liefert. Diese lassen sich aber nur auf wenigen (heute nicht mehr
      aktuellen) Maschinen in Hardware darstellen, und die Verarbeitung in Software dauert sehr lange.\\
      Als Alternative wird daher in JPEG (vor Standard JPEG-2000) die \emph{Diskrete Cosinus-Transformation}(DCT)
      verwendet, welche den selben Effekt wie die DFT erzielt, aber nur reelle Werte liefert.
	\item[Wo liegen die Informationsmaxima?]
      Die Informationsmaxima liegen am Rand, da die höherfrequenten Anteile dort liegen.
	\item[Nenne die Formel der DFT]
	\item[Fouriertransformation im $\R^2$]
      Die Fouriertransformation im zweidimensionalen Raum ist entsprechend die Variante für Bilder.\\
      Hierbei hat man also Punkte $p = (x, y) \in \Z^2$, entsprechen betrachtet man auch die Fouriertransformation
      in zwei unterschiedlichen Richtungen (in $x$- und in $y$-Richtung). Formal übersetzt ergibt sich also die
      Fouriertransformation im $\R^2$ als
      $$ \hat f(\xi) = \int_\R^2 f(t)e^{-i \xi^T t}\INT{t}\quad, \xi,t\in\R^2 $$
\end{description}
\subsection{Faltung}
\begin{description}
  \item[Was ist formal eine Faltung?]
    Formal gesehen ist eine Faltung ein Integral. Dieses ist für Funktionen $f, g\in\R^n \to \C$ definiert als
    $$ (f*g) = \int_{\R^n} f(\cdot - t)g(t)\INT{t}. $$
    Prinzipiell ist dieser Ausdruck definiert für alle Funktionen, die für \emph{fast alle} $x\in\R$ wohldefiniert
    sind. In unserem eingeschränkten Kontext der integrierbaren  Funktionen ist dies automatisch erfüllt.
  \item[Wie kann man sich so eine Faltung graphisch vorstellen?]
    Man kann sich das so vorstellen, dass man eine Funktion über die zweite \emph{schiebt} und dort die 
    Übereinstimmung berechnet. Bei einer hohen Übereinstimmung zwischen den Funktionen wird dieses Integral einen
    hohen Wert liefern, während sich bei wenig Übereinstimmung nur ein kleiner Wert ergibt.
  \item[Was bringt so eine Faltung?]
    Das im letzten Punkt beschriebene Verhalten der Faltung kann man sich z.B. in der Texterkennung zu nutze machen.
    Dabei prüft man z.B., ob es eine hohe Übereinstimmung zwischen einem Bild eines Buchstaben und dem vorliegenden
    Bild gibt, indem man über diese Bildteile die Faltung berechnet. Ein hoher Wert indiziert dabei, dass mit hoher
    Wahrscheinlichkeit der gesuchte Buchstabe vorliegt.\\
    Ganz allgemein ist dieses Prinzip die Grundlage von Filterungen.
\end{description}

\section{Filter}
\begin{description}
	\item[Was ist ein Filter?]
	\item[Wie sieht ein Filter allgemein aus?]
	\item[Wie sieht ein Tiefpass aus? Zeichne den Graph]
	\item[Forderungen an die Transferfunktion für Tiefpass]
	\item[Erkläre das Gibbs-Phänomen]
	\item[Welche Arten von Filter gibt es? Erklären Sie diese]
	\item[Zeichen Sie ein Schaltbild zu einem Filter (Addierer,Verzögerer, Multiplizierer)]
	\item[Welchen Filter zur Kantenerkennung?]
	\item[Was ist eine Impulsantwort?]
	\item[Spielt die Laufzeit von Filtern in der Praxis eine Rolle?]
\end{description}
\subsection{Was ist ein Gradientenfilter?}
\begin{description}
	\item[Wozu ist er gut?]
	\item[Wie funktioniert er?]
	\item[Wie sieht so ein Filter aus? (Gradient + Filtermatrix)]
	\item[Wie kommt man auf die Impulsantwort?]
	\item[Nachteile + mögliche Gegenmaßnahmen]
	\item[Wie sieht so ein Filter aus?]
\end{description}
\subsection{Filterbänke}
\begin{description}
	\item[Skizziere eine Filterbank und beschreibe den Vorgang]
	\item[Nenne 3 typische Filter]
	\item[Was ist die zentrale Eigenschaft von Filterbänken? Welche Voraussetzung muss dazu gelten?]
	\item[Kann man Filterbänke hierarchisch aufbauen?]
\end{description}
\section{Abtastsatz}
\begin{description}
	\item[Welche Eigenschaften müssen für eine Abtastung gelten?]
	\item[Erkläre den Shannonschen Abtastsatz]
	\item[Wie funktioniert Abtasten überhaupt?]
	\item[Was ist die kritische Abtastrate?]
	\item[Wie berechnet man die kritische Abtastrate?]
	\item[Was ist ein bandbeschränktes Signal?]
	\item[Was ist der Träger eines Signals?]
	\end{description}
\section{Hugh-Transformation}
\begin{description}
	\item[Was ist die Hugh-Transformation?]
	\item[Wie funktioniert sie?]
	\item[Was ermöglicht sie?]
	\item[Welcher Filter spielt dabei eine Rolle?]
	\item[Gibt es weitere Filter zur Kantenerkennung?]
\end{description}
\section{Muss noch zugeordnet werden}
\begin{description}
	\item[Erklären Sie die Heisenbergboxen]
\end{description}
\end{document}

