\documentclass[oneside]{scrbook}

\usepackage[utf8]{inputenc}
\usepackage[ngerman]{babel}
\usepackage{microtype}

\usepackage{scrhack} % Verschiedene Hacks für die Kompatibilität anderer Pakete mit KOMA-Script.

\usepackage{libertine}
\usepackage{amsmath,amssymb,amsthm}
\usepackage[libertine]{newtxmath} % Libertine Math.
\usepackage{commath}

\usepackage{wasysym}

\usepackage[T1]{fontenc}

% Schusterjungen und Hurenkinder
\widowpenalties 3 10000 10000 100
\clubpenalties 3 10000 10000 100
\displaywidowpenalty = 10000

\usepackage[german=guillemets]{csquotes}
\usepackage{booktabs}

\usepackage{xcolor}
\usepackage[pdftex]{graphicx}
\usepackage{tikz}
\usetikzlibrary{shapes}
\usepackage{hyperref}

\allowdisplaybreaks % Allows pagebreaking of aligned equations

\newtheoremstyle{myplain}% name
  {\topsep}% Space above
  {\topsep}% Space below
  {\itshape}% Body font
  {}% Indent amount
  {\bfseries\sffamily}% Theorem head font
  {}%Punctuation after theorem head
  {.5em}%Space after theorem head
  {}% theorem head spec

\newtheoremstyle{mydefinition}% name
  {\topsep}% Space above
  {\topsep}% Space below
  {\normalfont}% Body font
  {}% Indent amount
  {\bfseries\sffamily}% Theorem head font
  {}%Punctuation after theorem head
  {.5em}%Space after theorem head
  {}% theorem head spec

\newtheoremstyle{myremark}% name
  {\topsep}% Space above
  {\topsep}% Space below
  {\normalfont}% Body font
  {}% Indent amount
  {\itshape\sffamily}% Theorem head font
  {}%Punctuation after theorem head
  {.5em}%Space after theorem head
  {}% theorem head spec

\theoremstyle{myplain}
\newtheorem{proposition}{Satz}[chapter]

\theoremstyle{mydefinition}
\newtheorem{definition}[proposition]{Definition}
\newtheorem{example}[proposition]{Beispiel}
\newtheorem{remark}[proposition]{Bemerkung}

\newcommand{\N}{\mathbb{N}}
\newcommand{\R}{\mathbb{R}}
\newcommand{\Z}{\mathbb{Z}}
\newcommand{\C}{\mathbb{C}}
\newcommand{\T}{\mathbb{T}}

\DeclareMathOperator{\sinc}{sinc}

\newcommand{\TODO}[1]{{\color{red}\textbf{\textsf{TODO:}}~#1}}
\usepackage[pdftex]{graphicx}  
\begin{document}

\title{Bildverarbeitung}
\subtitle{Fragen}
\author{Manuel Pauli\and{}Sebastian Schweikl}

\maketitle
\section{Allgemein} 
\begin{description}
	\item[Erkläre Operationen: Skalierung, Translation]
	\item[Erkläre das Prinzip einer Lochkamera]
\end{description}	
\section{Fourier}
\subsection{Fouriertransformation}
\begin{description}
	\item[Nenne die Gleichung]
	\item[Wie sieht das Ergebnis der FT bei einer reelwertigen Funktion aus?]
	\item[Wie sieht die inverse Fouriertransformation aus?]
	\item[Zusammenhang zwischen Faltung und Fouriertransformation]
	\item[Erkläre den Weg zur FFT]
	\item[Wie schnell ist die FFT?]
	\item[Wie funktioniert das .jpg Kompressionsverfahren mit Hilfe der DFT?]
	\item[Wo liegen die Informationsmaxima?]
	\item[Nenne die Formel der DFT]
	\item[Fouriertransformation im $\R^2$]
\end{description}
\subsection{Faltung}
\section{Filter}
\begin{description}
	\item[Was ist ein Filter?]
	\item[Wie sieht ein Filter allgemein aus?]
	\item[Wie sieht ein Tiefpass aus? Zeichne den Graph]
	\item[Forderungen an die Transferfunktion für Tiefpass]
	\item[Erkläre das Gibbs-Phänomen]
	\item[Welche Arten von Filter gibt es? Erklären Sie diese]
	\item[Zeichen Sie ein Schaltbild zu einem Filter (Addierer,Verzögerer, Multiplizierer)]
	\item[Welchen Filter zur Kantenerkennung?]
	\item[Was ist eine Impulsantwort?]
	\item[Spielt die Laufzeit von Filtern in der Praxis eine Rolle?]
\end{description}
\subsection{Was ist ein Gradientenfilter?}
\begin{description}
	\item[Wozu ist er gut?]
	\item[Wie funktioniert er?]
	\item[Wie sieht so ein Filter aus? (Gradient + Filtermatrix)]
	\item[Wie kommt man auf die Impulsantwort?]
	\item[Nachteile + mögliche Gegenmaßnahmen]
	\item[Wie sieht so ein Filter aus?]
\end{description}
\subsection{Filterbänke}
\begin{description}
	\item[Skizziere eine Filterbank und beschreibe den Vorgang]
	\item[Nenne 3 typische Filter]
	\item[Was ist die zentrale Eigenschaft von Filterbänken? Welche Voraussetzung muss dazu gelten?]
	\item[Kann man Filterbänke hierarchisch aufbauen?]
\end{description}
\section{Abtastsatz}
\begin{description}
	\item[Welche Eigenschaften müssen für eine Abtastung gelten?]
	\item[Erkläre den Shannonschen Abtastsatz]
	\item[Wie funktioniert Abtasten überhaupt?]
	\item[Was ist die kritische Abtastrate?]
	\item[Wie berechnet man die kritische Abtastrate?]
	\item[Was ist ein bandbeschränktes Signal?]
	\item[Was ist der Träger eines Signals?]
	\end{description}
\section{Hugh-Transformation}
\begin{description}
	\item[Was ist die Hugh-Transformation?]
	\item[Wie funktioniert sie?]
	\item[Was ermöglicht sie?]
	\item[Welcher Filter spielt dabei eine Rolle?]
	\item[Gibt es weitere Filter zur Kantenerkennung?]
\end{description}
\section{Muss noch zugeordnet werden}
\begin{description}
	\item[Erklären Sie die Heisenbergboxen]
\end{description}
\end{document}
